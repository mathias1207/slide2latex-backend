\documentclass{beamer}
\usepackage{tcolorbox}
\tcbuselibrary{theorems}
\newtcolorbox{myvulga}[1]{colback=red!5!white,colframe=red!75!black,fonttitle=\bfseries,title=#1}
\newtheorem{remark}{Remarque}
\begin{document}
\begin{frame}{Slide 1}
Operating Systems (234123)\\Processes \& Signals\\Dan Tsafrir \\2024-06-03, 2024-06-10\\OS (234123) - processes \& signals\\1
\end{frame}
\begin{frame}{Slide 2}
What’s a process\\• An implementation of the abstract machine concept\\– Which we discussed in the previous lecture\\• A running instance of an executable, invoked by a user\\– Can have multiple independent processes of the same executable\\• A schedulable entity, on the CPU\\– OS decides which of these entities gets to run on a CPU core, and when\\• Sometimes called\\– Task or job\\• The OS kernel is neither a process nor a schedulable entity\\– Rather, it’s a set of procedures executing in response to events \\(≈ interrupts)\\– Albeit sometimes the OS runs some code within schedulable entities\\• But then we prefer not to refer to these entities as “processes”, \\which correspond to user programs; we may refer to them as \\“kernel threads“ instead\\OS (234123) - processes \& signals\\2
\end{frame}
\begin{frame}{Slide 3}
Process address space is contiguous\\OS (234123) - processes \& signals\\3\\0x0000…0000\\max\_address\\(e.g. 0x…ffff)\\address space\\code\\(text segment)\\static data\\(data segment)\\heap\\(dynamic allocated mem)\\stack\\(dynamic allocated mem)\\program counter\\stack pointer
\end{frame}
\begin{frame}{Slide 4}
Process states\\OS (234123) - processes \& signals\\4\\zombie\\ready\\running\\waiting\\created\\finished\\or killed\\scheduled\\preempted\\process is\\runnable\\process is\\sleeping\\termination\\status\\collected
\end{frame}
\begin{frame}{Slide 5}
• The OS maintains a “state” \\for every process\\– Encapsulated in a  PCB\\• In Linux\\– Called a “process descriptor”\\– Of type task\_struct (C struct)\\– Has O(100) fields\\• Used in context switches\\– Updated upon preemption\\– Loaded upon resumption\\• Question\\– Can a process access its own \\PCB?\\•\\PID (process ID)\\•\\UID (user ID)\\•\\Pointer to address space\\•\\Registers\\•\\Scheduling priority\\•\\Resources usage limits (e.g., \\memory, CPU, num of open files)\\•\\Resources consumed\\•\\State (previous slide)\\•\\Current/present working \\directory (pwd=cwd)\\•\\Open files table\\•\\…\\OS (234123) - processes \& signals\\5\\Process control block (“PCB”)\\process attributes \\in PCB
\end{frame}
\begin{frame}{Slide 6}
Context switching (in runnable state)\\OS (234123) - processes \& signals\\6\\process 1\\process 2\\kernel\\running\\running\\ready\\ready\\ready\\running\\save state (notably, registers) in PCB1\\load state from PCB2\\save state in PCB2\\syscall/interrupt\\syscall/interrupt\\load state from PCB1
\end{frame}
\begin{frame}{Slide 7}
Process creation \& termination\\• One process (the “parent”) can create another (the “child”)\\– A new PCB is allocated and initialized\\– Homework: run ‘ps auxwww’ in the shell; PPID is the parent’s PID\\• In POSIX, child process inherits most of parent’s attributes\\– UID, open files (should be closed if unneeded; why?), cwd, etc.\\• While executing, PCB moves between different queues\\– According to state change graph \\– Queues: runnable, sleep/wait for event i (i=1,2,3…)\\• After a process dies (exit()s / interrupted), it becomes a zombie\\– Parent uses wait* syscall to clear zombie from the system (why?)\\– Wait syscall family: wait, waitpid, waitid, wait3, wait4; example:\\– pid\_t wait4(pid\_t, int *wstatus, int options, struct rusage *rusage); \\• Parent can sleep/wait for its child to finish or run in parallel\\– wait*() will block unless WNOHANG given in ‘options’\\– Homework: read ‘man 2 wait’\\OS (234123) - processes \& signals\\7
\end{frame}
\begin{frame}{Slide 8}
int main(int argc, char *argv[])\\\{\\  int pid = fork();\\  if( pid==0 ) \{ \\   //\\   // child\\      //\\      printf(“parent=\%d son=\%d\textbackslash\{\}n”,\\             getppid(), getpid());\\  \}\\  else if( pid > 0 ) \{\\      //\\      // parent\\      //\\      printf(“parent=\%d son=\%d\textbackslash\{\}n”,\\             getpid(), pid);\\  \}\\  else \{ // print string associated\\         // with errno   \\      perror(“fork() failed”); \\  \}\\  return 0;\\\}\\• fork() initializes a new PCB\\– Based on parent’s value\\– PCB added to runnable queue\\• Now there are 2 processes\\– At same execution point\\• Child’s new address space \\– Complete copy of parent’s \\space, with one difference…\\• fork() returns twice\\– At the parent, with pid>0\\– At the child, with pid=0\\• What’s the printing order?\\• ‘errno’ – a global variable\\– Holds error num of last syscall\\OS (234123) - processes \& signals\\8\\fork() – spawn a child process
\end{frame}
\begin{frame}{Slide 9}
System call errors\\// int errno = number of last system call error.\\// Errors aren’t zero. (If you want to test value of\\// errno after a system call, need to zero it before.)\\\#include <errno.h> // see man 3 errno\\// const char * const sys\_errlist[];\\// char* strerror(int errnum) \{ \\// \\// check errnum is in range\\// \\return sys\_errlist[errnum];\\// \}\\\#include <string.h>\\//  void perror(const char *prefix);\\// prints: “\%s: \%s\textbackslash\{\}n” , prefix, sys\_errlist[errno]\\\#include <stdio.h>\\OS (234123) - processes \& signals\\9
\end{frame}
\begin{frame}{Slide 10}
exec*() – replace current process image\\• To start an entirely new program\\– Use the exec*() syscall family; for example:\\• int execv(const char *progamPath, char *const argv[]);\\– Homework: read ‘man execv’\\• Semantics\\– Stops the execution of the invoking process\\– Loads the executable ‘programPath’ \\– Starts ‘programPath’, with ‘argv’ as its argv\\– Never returns (unless fails)\\– Replaces the new process; doesn’t create a new process\\• In particular, PID and PPID are the same before/after exec*() \\OS (234123) - processes \& signals\\10
\end{frame}
\begin{frame}{Slide 11}
Simplistic UNIX shell loop example\\OS (234123) - processes \& signals\\11\\int main(int argc, char *argv[])\\\{\\  for(;;) \{\\      int stat;\\      char **argv;\\      char *c = readNextCom(\&argv);\\      int pid = fork();\\      if( pid < 0 ) \{\\          perror(“fork failed”);\\      \}\\      else if( pid==0 ) \{ // child\\          execv(c, argv);\\          perror(“execv failed”);\\      \}\\      else \{ // parent\\          if( wait(\&stat) < 0 )\\              perror(“wait failed”);\\          else\\              chkStatus(pid,stat);\\       release(argv);\\      \}\\  \}\\  return 0;\\\}\\void chkStatus(int pid, int stat)\\\{\\  if( WIFEXITED(stat) ) \{\\    printf(“\%d exit code=\%d\textbackslash\{\}n”,\\          pid, WEXITSTATUS(stat));\\  \}\\  else if( WIFSIGNALED(stat) ) \{\\    // the topic we’re going\\    // to learn next\\    printf(“\%d died on signal=\%d\textbackslash\{\}n”,\\          pid, WTERMSIG(stat));\\  \}\\  else if \\     // a few more options…\\\}\\Finished \\here
\end{frame}
\begin{frame}{Slide 12}
Who wait()-s for an “orphan” process?\\• POSIX specification says:\\– “If a parent process terminates without waiting for all of its child \\processes to terminate, the remaining child processes shall be assigned \\a new parent process ID corresponding to an implementation-defined \\system process”\\• https://pubs.opengroup.org/onlinepubs/9699919799/functions/w\\ait.html\\• Linux manual says:\\– “init […becomes] the parent of all processes whose natural parents \\have died, and it is responsible for reaping those when they die. […] \\init expects to have a process id of 1”\\• https://linux.die.net/man/8/init\\– “If a parent process terminates, then its zombie children (if any) are \\adopted by init”\\• https://linux.die.net/man/2/wait\\OS (234123) - processes \& signals\\12
\end{frame}
\begin{frame}{Slide 13}
POSIX SIGNALS\\OS-supported asynchronous notifications\\OS (234123) - interrupts \& signals\\13
\end{frame}
\begin{frame}{Slide 14}
Reminder from the 1st lecture\\OS (234123) - intro\\14\\signals\\(privileged ops ≠\\super user)
\end{frame}
\begin{frame}{Slide 15}
What are signals \& signal handlers \\• Signal = notification “sent” to a process\\– To asynchronously notify it that some event occurred \\• When receiving a signal\\– The process stops whatever it is doing \& handles it\\• Default signal handling action\\– Either die or ignore (depends on the type of the signal)\\• The process can configure how it handles most signals\\– Different signals can have a different handlers, \\and they can be temporarily blocked/unblocked = “masked”/”unmasked”\\• Except for 2 signals (well, actually 3 – discussed shortly)\\• Signals have names and numbers standardized by POSIX\\– Do ‘man 7 signal’ in shell/Google for a listing/explanation of all signals\\– For example: http://man7.org/linux/man-pages/man7/signal.7.html ,\\https://pubs.opengroup.org/onlinepubs/9699919799/basedefs/signal.h.html\\– HOMEWORK: take a few minutes to quickly survey all signals\\OS (234123) - interrupts \& signals\\15
\end{frame}
\begin{frame}{Slide 16}
Silly example\\\#include <signal.h>\\\#include <stdio.h>\\\#include <stdlib.h>\\void sigfpe\_handler(int signum) \{\\    fprintf(stderr,"I divided by zero (sig=\%d)!\textbackslash\{\}n“,\\            signum);    // prints SIGFPE’s const value\\    exit(EXIT\_FAILURE); // what happens if not exiting?\\\}\\int main() \{\\    signal(SIGFPE, sigfpe\_handler);\\    int x = 1/0; // processor interrupt, then OS signal\\    return 0;\\\}\\OS (234123) - interrupts \& signals\\16
\end{frame}
\begin{frame}{Slide 17}
Another silly example\\\#include <signal.h>\\\#include <stdio.h>\\\#include <stdlib.h>\\void sigint\_handler(int signum) \{\\    printf("I’m disregarding your crtl-c!\textbackslash\{\}n");\\\}\\int main() \{\\    // when pressing ctrl-c in the shell \\    // => SIGINT is delivered to foreground process\\    // (who makes this happen?)\\    signal(SIGINT,sigint\_handler); \\    for(;;) \{ /*endless loop*/ \}\\    return 0;\\\}\\OS (234123) - interrupts \& signals\\17
\end{frame}
\begin{frame}{Slide 18}
Another silly example\\<0>dan@csa:\textasciitilde{}\$ ./a.out \\\# here I clicked ctrl-c => delivered SIGINT\\I’m disregarding your crtl-c!\\\# here I clicked ctrl-c again => delivered SIGINT\\I’m disregarding your crtl-c!\\\# here I clicked ctrl-z => delivered SIGSTOP, must obey\\[1]+  Stopped                 ./a.out\\<148>dan@csa:\textasciitilde{}\$ ps\\  PID TTY          TIME CMD\\10148 pts/19   00:00:00 bash\\21709 pts/19   00:00:12 a.out\\21710 pts/19   00:00:00 ps\\<0>dan@csa:\textasciitilde{}\$ kill -9 21709   \# 9=SIGKILL, must obey\\[1]+  Killed                  ./a.out\\<0>dan@csa:\textasciitilde{}\$\\OS (234123) - interrupts \& signals\\18
\end{frame}
\begin{frame}{Slide 19}
Another silly example\\<0>dan@csa:\textasciitilde{}\$ ./a.out \\\# here I clicked ctrl-c (=> deliver SIGINT)\\I'm ignoring your crtl-c!\\\# here I clicked ctrl-c again (=> deliver SIGINT)\\I'm ignoring your crtl-c!\\\# here I clicked ctrl-z => deliver SIGSTP, can’t ignore\\[1]+  Stopped                 ./a.out\\<148>dan@csa:\textasciitilde{}\$ ps\\  PID TTY          TIME CMD\\10148 pts/19   00:00:00 bash\\21709 pts/19   00:00:12 a.out\\21710 pts/19   00:00:00 ps\\<0>dan@csa:\textasciitilde{}\$ kill -9 21709   \# 9=SIGKILL, can’t ignore\\[1]+  Killed                  ./a.out\\<0>dan@csa:\textasciitilde{}\$\\OS (234123) - interrupts \& signals\\19\\When I click ctrl-c, the OS gets an interrupt \\from the keyboard, which the OS then \\translates into a SIGINT signal delivered to \\the relevant process.
\end{frame}
\begin{frame}{Slide 20}
Notice\\• Argument for the ‘kill’ shell utility \\– Any signal (not just 9=kill)\\• kill -9 <pid>\\• kill -s KILL <pid>\\• kill -s SIGKILL <pid>\\• There are 2+1=3 signals that a process can’t ignore\\– SIGKILL = terminate the receiving process\\– SIGSTOP = suspend the receiving process (make it sleep)\\• Is the affect of SIGSTOP reversible?\\– Yes, when you send to the process SIGCONT\\• SIGCONT can’t be ignored either…\\– A SIGSTOP-ed process *will* continue\\– But process can set a handler for it, which will be invoked immediately  \\when the process gets hit by the SIGCONT and is resumed as a result\\• What can you do with SIGSTOP/SIGCONT?\\OS (234123) - interrupts \& signals\\20
\end{frame}
\begin{frame}{Slide 21}
Job control\\• In the shell, assume we run a program ‘loop’ that does this\\– int main() \{ while(1); return 0; \}\\• As noted, clicking ctrl-z in the shell\\– Will make ‘loop’ sleep\\• Subsequently, invoking ‘fg’ in the shell\\– Will wake ‘loop’ up in the foreground (meaning, the shell sleeps, and \\any typed input is directed to ‘loop’)\\• Alternatively, invoking ‘bg’ in the shell\\– Will wake ‘loop’ up in the background (as is we executed it with “\&”, \\so shell becomes operational, and typed input goes to the shell)\\• Simplifying lie I told\\– For reasons related to job control, actually, ctrl-z \\• Generates STGTSTP, not SIGSTOP\\– There are subtle differences between the two, which we won’t learn\\OS (234123) - processes \& signals\\21
\end{frame}
\begin{frame}{Slide 22}
This is how a signal is truly ignored\\\#include <signal.h>\\\#include <stdio.h>\\\#include <stdlib.h>\\int main() \{\\    // Before, we used sigint\_handler as the arg.\\    // Now, we use the SIG\_IGN macro, which means\\    // no handler will be called. \\    // There’s also SIG\_DFL, to restore the default\\    // behavior\\    signal(SIGINT, SIG\_IGN); \\    for(;;) \{ /*endless loop*/ \}\\    return 0;\\\}\\OS (234123) - interrupts \& signals\\22
\end{frame}
\begin{frame}{Slide 23}
Example – ask a running daemon \\how much “work” it did thus far\\void do\_work() \{ for(int i=0; i<10000000; i++); \}\\int g\_count=0; // counts num. of times do\_work was invoked\\void sigusr\_handler(int signum) \{\\    printf("Work done so far: \%d\textbackslash\{\}n", g\_count);\\\}\\int main() \{\\    signal(SIGUSR1,sigusr\_handler);\\    for(;;) \{ do\_work(); g\_count++; \}\\    return 0;\\\}\\OS (234123) - interrupts \& signals\\23\\• A “daemon” is\\– Background process, not controlled interactively by user\\– In shell: nohup <command> \& (see: https://linux.die.net/man/1/nohup)\\– Daemon name typically ends with “d” (e.g., sshd, syslogd, swapd)
\end{frame}
\begin{frame}{Slide 24}
<0>dan@csa:\textasciitilde{}\$ ./a.out \&\\[1] 23998\\\# recall: kill utility also accepts strings as signals\\<0>dan@csa:\textasciitilde{}\$ kill –s USR1 23998\\Work done so far: 626\\<0>dan@csa:\textasciitilde{}\$ kill –s USR1 23998\\Work done so far: 862\\<0>dan@csa:\textasciitilde{}\$ kill –s USR1 23998\\Work done so far: 1050\\<0>dan@csa:\textasciitilde{}\$ kill –s KILL 23998\\[1]+  Killed                  ./a.out\\OS (234123) - interrupts \& signals\\24\\Example – ask a running daemon \\how much “work” it did thus far
\end{frame}
\begin{frame}{Slide 25}
Enumerate signals\\• SIGSEGV, SIGBUSS, SIGILL, SIGFPE\\– ILL = illegal instruction (trying to invoke privileged instruction)\\– SEGV = segmentation violation (illegal memory ref, e.g., outside an array) \\– BUS = dereference invalid address (null/misaligned, assume it’s like SEGV)\\– FPE = floating point exception (despite name, actually all arithmetic \\errors, nor just floating point; example: divide by zero)\\– These are driven by the associated (HW) interrupts\\• The OS gets the associated interrupt\\• The OS interrupt handler sees to it that the misbehaving process gets \\the associated signal\\• The default signal handler for these signals: core dump + die\\• SIGCHLD\\– Parent gets it whenever fork()ed child terminates or is SIGSTOP-ed\\• SIGALRM\\– Get a signal after some specified time\\– Set by system calls: alarm(2) \& setitimer(2) (homework: read man)\\OS (234123) - interrupts \& signals\\25
\end{frame}
\begin{frame}{Slide 26}
Enumerate signals\\• SIGTRAP\\– When debugging / single-stepping a process\\– E.g., can be delivered upon each instruction\\• SIGUSR1, SIGUSR2\\– User decides the meaning (e.g., see our daemon example)\\• SIGXCPU\\– Delivered when a process used up more CPU then its soft-limit allows\\– Soft/hard limits are set by the system call: setrlimit()\\– Soft-limits warn the process its about to exceed the hard-limit\\– Exceeding the hard-limit => SIGKILL will be delivered\\• SIGPIPE\\– Write to pipe with no readers (we’ll learn about pipes later, for the \\time being think about the shell’s pipe: “|”)\\OS (234123) - interrupts \& signals\\26
\end{frame}
\begin{frame}{Slide 27}
Enumerate signals\\• SIGIO\\– Can configure file descriptors such that a signal will be delivered \\whenever some I/O is ready\\– Typically makes sense when also configuring the file descriptors to be \\“non blocking” \\• E.g., when read()ing from a non-blocking file descriptor, the system \\call immediately returns to user if there’s currently nothing to read\\• In this case, errno will be set to EAGAIN = EWOULDBLOCK\\• And a few more\\– man 7 signal\\– http://man7.org/linux/man-pages/man7/signal.7.html\\OS (234123) - interrupts \& signals\\27
\end{frame}
\begin{frame}{Slide 28}
Signal vs. interrupts\\OS (234123) - interrupts \& signals\\28\\interrupts\\signals\\Who triggers them?\\Who defines their meaning?\\Hardware: \\CPU cores (sync) \& \\other devices (async) \\Software (OS), \\HW is unaware\\Who handles them?\\Who (un)blocks them?\\OS\\processes\\When do they occur?\\Both synchronously \\\& asynchronously\\Likewise, but, \\technically, invoked \\when returning \\from kernel to user
\end{frame}
\begin{frame}{Slide 29}
Signal system calls – sending\\• int kill(pid\_t pid, int sig)\\– (Not the shell utility, the actual system call)\\– Allows a process to send a signal to another process (or to itself)\\• Homework: How?\\– man 2 kill – http://linux.die.net/man/2/kill\\• “2” is for system calls\\• “1” is for shell utilities\\OS (234123) - interrupts \& signals\\29
\end{frame}
\begin{frame}{Slide 30}
Signal system calls – (un)blocking\\• Signals can be asynchronous => might lead to ”race conditions”\\– Therefore, as noted, all signals (except kill/stop) can be blocked\\– Like how OS disables/enables interrupt\\• How\\– The PCB maintains a set of currently blocked signals\\– Which can be manipulated by users via the following syscall\\• int sigprocmask(int how, const sigset\_t *set, sigset\_t *oldset)\\– ‘how’ = SIG\_BLOCK (+=), SIG\_UNBLOCK (-=), SIG\_SETMASK (=)\\– ‘set’ can be maipulated with sigset ops sigetmptyset(sigset\_t *set),    \\sigfillset(sigset\_t *), sigaddset(sigset\_t *set, int signum), \\sigismember(sigset\_t *set, int signum)\\• Manual\\– man 2 sigprocmask – http://linux.die.net/man/2/sigprocmask\\– man 3 sigsetops – https://linux.die.net/man/3/sigsetops\\OS (234123) - interrupts \& signals\\30
\end{frame}
\begin{frame}{Slide 31}
(Un)blocking example\\OS (234123) - processes \& signals\\31\\Record\_t db[N];\\void my\_handler(int signum) \{\\  /* may read/update db */\\\}\\int main(int argc, char *argv[]) \\\{\\  sigset\_t mask, orig\_mask;\\  sigemptyset(\&mask);\\  sigaddset(\&mask, SIGTERM); \\  signal(SIGTERM, my\_handler);\\  for(;;) \{\\    char *cmd = read\_command();\\    sigprocmask(SIG\_BLOCK, \&mask, \&orig\_mask);\\    // do stuff that may read/update db…\\    sigprocmask(SIG\_SETMASK, \&orig\_mask, NULL);\\  \}\\  return 0;\\\}\\Recall that every syscall might fail;\\the example ignores this for brevity
\end{frame}
\begin{frame}{Slide 32}
Signal system calls – control \& more info\\• Additional info about signal \& fine-grain control \\– over how signals operate is provided via the following syscall\\• int sigaction(int signum, \\ \\const struct sigaction *act, \\        struct sigaction *oldact)\\– man 2 sigaction – http://linux.die.net/man/2/sigaction \\(homework: read it)\\OS (234123) - interrupts \& signals\\32
\end{frame}
\begin{frame}{Slide 33}
Signals interact with other system calls\\• ssize\_t read(int fd, void *buf, size\_t count);\\– What happens if getting signal while read()ing?\\– The read system call returns -1, and it sets the global variable ‘errno’ \\to hold EINTR\\– An example whereby read() might fail and user should simply retry\\OS (234123) - interrupts \& signals\\33\\int readn( int sockfd /*learn later*/, char *ptr, int nbytes )\\\{\\  int nleft = nbytes;\\    \\  while( nleft > 0 ) \{ // ‘read’ is typically done in a loop (why?)\\      int nread = read(sockfd, ptr, nleft);\\      if( (nread == -1) \&\& (errno != EINTER) ) \{\\          fprintf(stderr, “read failed, errno=\%m", errno);\\          return -1;\\      \}\\      else if( nread == 0 )\\         break; /*EOF*/\\        \\      nleft -= nread;\\      ptr   += nread;\\  \}\\  return nbytes - nleft;\\\}
\end{frame}
\end{document}