% !TEX program = xelatex
\documentclass[12pt]{report}
\usepackage[utf8]{inputenc}
\usepackage[english]{babel}

\usepackage{tcolorbox}
\usepackage{fontawesome5}
\usepackage{listings}
\usepackage{algorithm2e}
\usepackage{amsmath,amsfonts,amssymb,graphicx,float}
\usepackage{xcolor}
\usepackage{parskip}

% Configuration des listings
\lstset{
    basicstyle=\ttfamily\small,
    breaklines=true,
    frame=single,
    numbers=left,
    numberstyle=\tiny,
    numbersep=5pt,
    showstringspaces=false,
    language=C,
    keepspaces=true,
    columns=flexible,
    literate={*}{\textasteriskcentered}1
              {\_}{\_}1
              {-}{-}1,
    escapechar=@,
    texcl=true
}

% Configuration des tcolorbox
\tcbuselibrary{most}
\newtcolorbox{mybox}[1][]{
    colback=white,
    colframe=black,
    coltitle=black,
    title=#1,
    fonttitle=\bfseries
}

% Protection des underscores dans le texte
\usepackage{underscore}

% Configuration des marges
\usepackage[left=2.5cm,right=2.5cm,top=2.5cm,bottom=2.5cm]{geometry}

\graphicspath{{./images/}}
\title{osj}
\author{}

\begin{document}
\maketitle
\tableofcontents
\newpage

\section{Introduction}
% Cours : osj
Process creation & termination
• One process (the “parent”) can create another (the “child”)
– A new PCB is allocated and initialized
– Homework: run ‘ps auxwww’ in the shell; PPID is the parent’s PID
• In POSIX, child process inherits most of parent’s attributes
– UID, open files (should be closed if unneeded; why?), cwd, etc.
• While executing, PCB moves between different queues
– According to state change graph 
– Queues: runnable, sleep/wait for event i (i=1,2,3…)
• After a process dies (exit()s / interrupted), it becomes a zombie
– Parent uses wait* syscall to clear zombie from the system (why?)
– Wait syscall family: wait, waitpid, waitid, wait3, wait4; example:
– pid\_t wait4(pid\_t, int *wstatus, int options, struct rusage *rusage); 
• Parent can sleep/wait for its child to finish or run in parallel
– wait*() will block unless WNOHANG given in ‘options’
– Homework: read ‘man 2 wait’
OS (234123) - processes & signals
7

int main(int argc, char *argv[])
{
  int pid = fork();
  if( pid==0 ) { 
   //
   // child
      //
      printf(“parent=%d son=%d\n”,
             getppid(), getpid());
  }
  else if( pid > 0 ) {
      //
      // parent
      //
      printf(“parent=%d son=%d\n”,
             getpid(), pid);
  }
  else { // print string associated
         // with errno   
      perror(“fork() failed”); 
  }
  return 0;
}
• fork() initializes a new PCB
– Based on parent’s value
– PCB added to runnable queue
• Now there are 2 processes
– At same execution point
• Child’s new address space 
– Complete copy of parent’s 
space, with one difference…
• fork() returns twice
– At the parent, with pid>0
– At the child, with pid=0
• What’s the printing order?
• ‘errno’ – a global variable
– Holds error num of last syscall
OS (234123) - processes & signals
8
fork() – spawn a child process


\end{document}