\documentclass[12pt]{article}
\usepackage[utf8]{inputenc}
\usepackage[french]{babel}

\usepackage{tcolorbox}
\usepackage{fontawesome5}
\usepackage{listings}
\usepackage{amsmath}
\usepackage{xcolor}
\usepackage{geometry}

\% Configuration minimale de tcolorbox
\tcbuselibrary{most}

\% Configuration minimale des listings
\lstset{breaklines=true, basicstyle=\ttfamily\small}

\% Configuration des marges
\geometry{margin=2.5cm}

\title{osj}
\author{}
\date{}

\begin{document}
\maketitle
\tableofcontents
\newpage

\% Cours : osj
Process creation \& termination
• One process (the “parent”) can create another (the “child”)
– A new PCB is allocated and initialized
– Homework: run ‘ps auxwww’ in the shell; PPID is the parent’s PID
• In POSIX, child process inherits most of parent’s attributes
– UID, open files (should be closed if unneeded; why?), cwd, etc.
• While executing, PCB moves between different queues
– According to state change graph 
– Queues: runnable, sleep/wait for event i (i=1,2,3…)
• After a process dies (exit()s / interrupted), it becomes a zombie
– Parent uses wait* syscall to clear zombie from the system (why?)
– Wait syscall family: wait, waitpid, waitid, wait3, wait4; example:
– pid\_t wait4(pid\_t, int *wstatus, int options, struct rusage *rusage); 
• Parent can sleep/wait for its child to finish or run in parallel
– wait*() will block unless WNOHANG given in ‘options’
– Homework: read ‘man 2 wait’
OS (234123) - processes \& signals
7

\begin{lstlisting}


\section*{Fiche Récapitulative}
\section{Process creation \& termination}
\subsection{Contenu du chapitre}
\begin{itemize}
\item Process creation \& termination
\item One process (the 'parent') can create another (the 'child')
\begin{itemize}
\item A new PCB is allocated and initialized
\item Homework: run 'ps auxwww' in the shell; PPID is the parent's PID
\end{itemize}
\item In POSIX, child process inherits most of parent's attributes
\begin{itemize}
\item UID, open files (should be closed if unneeded; why?), cwd, etc.
\end{itemize}
\item While executing, PCB moves between different queues
\begin{itemize}
\item According to state change graph
\item Queues: runnable, sleep/wait for event i (i=1,2,3...)
\end{itemize}
\item After a process dies (exit()s / interrupted), it becomes a zombie
\begin{itemize}
\item Parent uses wait* syscall to clear zombie from the system (why?)
\item Wait syscall family: wait, waitpid, waitid, wait3, wait4; example:
\item pid_t wait4(pid_t, int *wstatus, int options, struct rusage *rusage);
\end{itemize}
\item Parent can sleep/wait for its child to finish or run in parallel
\begin{itemize}
\item wait*() will block unless WNOHANG given in 'options'
\item Homework: read 'man 2 wait'
\end{itemize}
\end{itemize}

\end{document}