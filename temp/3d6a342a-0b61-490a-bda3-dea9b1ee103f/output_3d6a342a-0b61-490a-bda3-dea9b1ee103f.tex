\documentclass[12pt]{article}
\usepackage[utf8]{inputenc}
\usepackage[T1]{fontenc}
\usepackage[french]{babel}

\usepackage{tcolorbox}
\usepackage{fontawesome5}
\usepackage{listings}
\usepackage{amsmath}
\usepackage{xcolor}
\usepackage{geometry}
\usepackage{textcomp}
\DeclareUnicodeCharacter{00A0}{~}
\DeclareUnicodeCharacter{200B}{}

% Configuration simplifiée pour éviter les erreurs avec \hss
\lstset{
    basicstyle=\ttfamily\small,
    breaklines=true,
    breakatwhitespace=false,
    keepspaces=true,
    numbers=left,
    numberstyle=\tiny\color{gray},
    showspaces=false,
    showstringspaces=false,
    showtabs=false,
    tabsize=2
}

% Configuration minimale de tcolorbox
\tcbuselibrary{most}

% Ajouter cette commande pour les échappements
\newcommand{\escapesym}[1]{\texttt{\textbackslash{}#1}}

% Configuration des marges
\geometry{margin=2.5cm}

\title{datta}
\author{}
\date{}

\begin{document}
\maketitle
\tableofcontents
\newpage

\section{Introduction}

\sub\section{Objectif : Digitalisation}

Vous avez un signal réel et vous souhaitez l'adapter à un ordinateur. Cependant, il existe des contraintes. Le nombre de valeurs pour représenter l'ensemble du signal est fini, noté $N$. Chaque valeur ne peut pas prendre n'importe quelle valeur réelle, mais seulement l'une des $k$ valeurs quantifiées fixes, où $b = \log_2(k)$.

\begin{tcolorbox}[title={Intuition}]
La digitalisation consiste à convertir un signal continu en une représentation discrète qui peut être traitée par un ordinateur. Imaginez une peinture transformée en pixels, puis en niveaux de couleur limités.
\end{tcolorbox}

% FIGURE: Illustration de la digitalisation avec la Mona Lisa

\newpage

\section{Échantillonnage}

\sub\section{Concept de l'échantillonnage}

Le signal $\varphi(x)$ est décrit par ses valeurs pour tous les $x$. Cela nécessite une infinité de valeurs pour représenter $\varphi$. L'objectif est de trouver un ensemble fini de nombres $\hat{\varphi}_1, \ldots, \hat{\varphi}_N$ pour représenter approximativement $\varphi$.

\begin{tcolorbox}[title={Vulgarisation simple}]
L'échantillonnage est comme prendre des photos à intervalles réguliers d'un film continu pour créer une version simplifiée.
\end{tcolorbox}

% FIGURE: Illustration de l'échantillonnage

\sub\section{Stratégies d'échantillonnage}

Il n'y a pas d'unicité dans les stratégies d'échantillonnage et de reconstruction. Étant donné une stratégie de reconstruction et un critère de "meilleur", il faut trouver les "meilleurs" $\hat{\varphi}_1, \ldots, \hat{\varphi}_N$.

\begin{tcolorbox}[title={Fiche Récapitulative}]
\textbf{Objectif :} Représenter un signal continu par un ensemble fini de valeurs.

\textbf{Principe central :} Échantillonnage et quantification.

\vspace{0.4em}
\textbf{Points essentiels :}  
- Trouver un ensemble fini de valeurs $\hat{\varphi}_1, \ldots, \hat{\varphi}_N$
- Stratégies d'échantillonnage et de reconstruction
- Importance des critères d'optimisation

\vspace{0.4em}
\textbf{À retenir :}  
- Utiliser des tirets pour les listes au lieu de \begin{itemize}
- Insérer des \vspace{0.4em} entre les \sections pour l'espacement
- Mettre en gras (\textbf) les titres des \sections
\end{tcolorbox}

\newpage

\section{Critère $L^2$}

\sub\section{Erreur quadratique moyenne}

L'erreur quadratique moyenne totale (MSE) est définie par :

\begin{equation}
MSE = \int_0^1 (\varphi(x) - \hat{\varphi}(x))^2 dx
\end{equation}

L'erreur quadratique moyenne locale est donnée par :

\begin{equation}
MSE_i = \frac{1}{\Delta} \int_{I_i} (\varphi(x) - \hat{\varphi}_i)^2 dx
\end{equation}

\begin{tcolorbox}[title={Intuition}]
Le critère $L^2$ mesure la différence quadratique entre le signal original et le signal échantillonné, favorisant les petites erreurs.
\end{tcolorbox}

\newpage

\section{Critère $L^1$}

\sub\section{Déviation absolue moyenne}

La déviation absolue moyenne totale (MAD) est définie par :

\begin{equation}
MAD = \int_0^1 |\varphi(x) - \hat{\varphi}(x)| dx
\end{equation}

La déviation absolue moyenne locale est donnée par :

\begin{equation}
MAD_i = \frac{1}{\Delta} \int_{I_i} |\varphi(x) - \hat{\varphi}_i| dx
\end{equation}

\begin{tcolorbox}[title={Vulgarisation simple}]
Le critère $L^1$ se concentre sur la somme des différences absolues, ce qui est plus robuste aux grandes erreurs que le critère $L^2$.
\end{tcolorbox}

\newpage

\section{Exemples}

\sub\section{Exemple 1}

Pour $\varphi(x) = x$, l'échantillonnage $L^2$ et $L^1$ donne des résultats différents en termes de MSE et MAD.

% FIGURE: Graphique de l'exemple 1

\sub\section{Exemple 2}

Pour un signal défini par morceaux, les valeurs de MSE et MAD sont calculées pour $N = 2$ échantillons.

% FIGURE: Graphique de l'exemple 2

\begin{tcolorbox}[title={Fiche Récapitulative}]
\textbf{Objectif :} Comparer les critères $L^2$ et $L^1$ à travers des exemples.

\textbf{Principe central :} Différences entre les erreurs quadratiques et absolues.

\vspace{0.4em}
\textbf{Points essentiels :}  
- Calcul des erreurs pour différents types de signaux
- Importance du choix du critère selon le contexte

\vspace{0.4em}
\textbf{À retenir :}  
- Utiliser des tirets pour les listes au lieu de \begin{itemize}
- Insérer des \vspace{0.4em} entre les \sections pour l'espacement
- Mettre en gras (\textbf) les titres des \sections
\end{tcolorbox}

\newpage

\section{Échantillonnage dans le Monde Discret}

Il est possible de sous-échantillonner en passant de $N$ à $M$ échantillons. Les échantillons optimaux sont les moyennes sur les intervalles discrets.

% FIGURE: Illustration de l'échantillonnage dans le monde discret

\begin{tcolorbox}[title={Intuition}]
Dans le monde discret, l'échantillonnage est comme réduire la résolution d'une image tout en conservant l'essentiel de l'information.
\end{tcolorbox}
\end{itemize}}

\end{document}