\documentclass[12pt]{article}
\usepackage[utf8]{inputenc}
\usepackage[T1]{fontenc}
\usepackage[french]{babel}

\usepackage{tcolorbox}
\usepackage{fontawesome5}
\usepackage{listings}
\usepackage{amsmath}
\usepackage{xcolor}
\usepackage{geometry}
\usepackage{textcomp}
\DeclareUnicodeCharacter{00A0}{~}
\DeclareUnicodeCharacter{200B}{}

\% Configuration minimale de tcolorbox
\tcbuselibrary{most}

\% Configuration minimale des listings
\lstset{
    breaklines=true, 
    basicstyle=\ttfamily\small,
    inputencoding=utf8,
    extendedchars=true,
    literate={á}{\'a}1 {é}{\'e}1 {í}{\'i}1 {ó}{\'o}1 {ú}{\'u}1
             {à}{\`a}1 {è}{\`e}1 {ì}{\`i}1 {ò}{\`o}1 {ù}{\`u}1
             {ä}{\"a}1 {ë}{\"e}1 {ï}{\"i}1 {ö}{\"o}1 {ü}{\"u}1
             {â}{\^a}1 {ê}{\^e}1 {î}{\^i}1 {ô}{\^o}1 {û}{\^u}1
             {ç}{\c c}1
}

\% Configuration des marges
\geometry{margin=2.5cm}

\title{osj}
\author{}
\date{}

\begin{document}
\maketitle
\tableofcontents
\newpage

\% Cours : osj
\section{Processus de création et de terminaison}\subsection{Création et terminaison de processus}\begin{itemize}\item Un processus (le "parent") peut créer un autre (l'enfant)\item Un nouveau PCB est alloué et initialisé\item Devoir : exécuter 'ps auxwww' dans le shell ; PPID est le PID du parent\end{itemize}\subsection{Héritage des attributs du processus enfant}\begin{itemize}\item En POSIX, le processus enfant hérite de la plupart des attributs du parent\item UID, fichiers ouverts (doivent être fermés s'ils ne sont pas nécessaires ; pourquoi ?), cwd, etc.\end{itemize}\subsection{Déplacement du PCB entre les files d'attente}\begin{itemize}\item Pendant l'exécution, le PCB se déplace entre différentes files d'attente\item Selon le graphe de changement d'état\item Files d'attente : prêt, sommeil/en attente d'un événement i (i=1,2,3...)\end{itemize}\subsection{Processus zombie}\begin{itemize}\item Après la mort d'un processus (exit() / interrompu), il devient un zombie\item Le parent utilise l'appel système wait* pour supprimer le zombie du système (pourquoi ?)\item Famille d'appels système wait : wait, waitpid, waitid, wait3, wait4 ; exemple :\item pid\_t wait4(pid\_t, int *wstatus, int options, struct rusage *rusage);\end{itemize}\subsection{Attente du parent pour le processus enfant}\begin{itemize}\item Le parent peut attendre que son enfant termine ou s'exécute en parallèle\item wait*() bloquera sauf si WNOHANG est donné dans 'options'\item Devoir : lire 'man 2 wait'\end{itemize}

int main(int argc, char *argv[])
{
  int pid = fork();
  if( pid==0 ) { 
   //
   // child
      //
      printf(“parent=\%d son=\%d\n”,
             getppid(), getpid());
  }
  else if( pid > 0 ) {
      //
      // parent
      //
      printf(“parent=\%d son=\%d\n”,
             getpid(), pid);
  }
  else { // print string associated
         // with errno   
      perror(“fork() failed”); 
  }
  return 0;
}
• fork() initializes a new PCB
– Based on parent’s value
– PCB added to runnable queue
• Now there are 2 processes
– At same execution point
• Child’s new address space 
– Complete copy of parent’s 
space, with one difference…
• fork() returns twice
– At the parent, with pid>0
– At the child, with pid=0
• What’s the printing order?
• ‘errno’ – a global variable
– Holds error num of last syscall
OS (234123) - processes \& signals
8
fork() – spawn a child process


\section*{Fiche Récapitulative}
\section{Processus de création et de terminaison}

\subsection{Création et terminaison de processus}
\begin{itemize}
\item Un processus (le "parent") peut créer un autre (l'enfant)
\item Un nouveau PCB est alloué et initialisé
\item Devoir : exécuter 'ps auxwww' dans le shell ; PPID est le PID du parent
\end{itemize}

\subsection{Héritage des attributs du processus enfant}
\begin{itemize}
\item En POSIX, le processus enfant hérite de la plupart des attributs du parent
\item UID, fichiers ouverts (doivent être fermés s'ils ne sont pas nécessaires ; pourquoi ?), cwd, etc.
\end{itemize}

\subsection{Déplacement du PCB entre les files d'attente}
\begin{itemize}
\item Pendant l'exécution, le PCB se déplace entre différentes files d'attente
\item Selon le graphe de changement d'état
\item Files d'attente : prêt, sommeil/en attente d'un événement i (i=1,2,3...)
\end{itemize}

\subsection{Processus zombie}
\begin{itemize}
\item Après la mort d'un processus (exit() / interrompu), il devient un zombie
\item Le parent utilise l'appel système wait* pour supprimer le zombie du système (pourquoi ?)
\item Famille d'appels système wait : wait, waitpid, waitid, wait3, wait4 ; exemple :
\item pid\_t wait4(pid\_t, int *wstatus, int options, struct rusage *rusage);
\end{itemize}

\subsection{Attente du parent pour le processus enfant}
\begin{itemize}
\item Le parent peut attendre que son enfant termine ou s'exécute en parallèle
\item wait*() bloquera sauf si WNOHANG est donné dans 'options'
\item Devoir : lire 'man 2 wait'
\end{itemize}

\begin{tcolorbox}[title={title={Fiche Récapitulative}]
Le contenu des fiches récapitulatives ici...
\end{tcolorbox}

\end{document}