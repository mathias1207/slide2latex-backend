\documentclass[12pt]{article}
\usepackage[utf8]{inputenc}
\usepackage[T1]{fontenc}
\usepackage[french]{babel}

\usepackage{tcolorbox}
\usepackage{fontawesome5}
\usepackage{listings}
\usepackage{amsmath}
\usepackage{xcolor}
\usepackage{geometry}
\usepackage{textcomp}
\DeclareUnicodeCharacter{00A0}{~}
\DeclareUnicodeCharacter{200B}{}

% Configuration simplifiée pour éviter les erreurs avec \hss
\lstset{
    basicstyle=\ttfamily\small,
    breaklines=true,
    breakatwhitespace=false,
    keepspaces=true,
    numbers=left,
    numberstyle=\tiny\color{gray},
    showspaces=false,
    showstringspaces=false,
    showtabs=false,
    tabsize=2
}

% Configuration minimale de tcolorbox
\tcbuselibrary{most}

% Ajouter cette commande pour les échappements
\newcommand{\escapesym}[1]{\texttt{\textbackslash{}#1}}

% Configuration des marges
\geometry{margin=2.5cm}

\title{lojm}
\author{}
\date{}

\begin{document}
\maketitle
\tableofcontents
\newpage

\section{Création et terminaison des processus}

Un processus, appelé le "parent", peut en créer un autre, le "fils". Lors de cette création, un nouveau PCB (Process Control Block) est alloué et initialisé. Pour observer cela, vous pouvez exécuter la commande \texttt{ps auxwww} dans le shell, où le PPID représente le PID du parent.

Dans le système POSIX, le processus fils hérite de la plupart des attributs du parent, tels que l'UID, les fichiers ouverts (qui devraient être fermés si inutiles), le répertoire de travail courant, etc.

Pendant l'exécution, le PCB se déplace entre différentes files d'attente selon le graphe de changement d'état. Ces files d'attente incluent les états exécutable, en sommeil ou en attente d'un événement.

Lorsqu'un processus meurt (par \texttt{exit()} ou interruption), il devient un zombie. Le parent utilise l'appel système \texttt{wait*} pour nettoyer le zombie du système. La famille d'appels système \texttt{wait} inclut \texttt{wait}, \texttt{waitpid}, \texttt{wait3}, et \texttt{wait4}. Par exemple :

\begin{lstlisting}
pid_t wait4(pid_t, int *wstatus, int options, struct rusage *rusage);
\end{lstlisting}

Le parent peut dormir ou attendre que son enfant se termine ou s'exécute en parallèle. L'appel \texttt{wait*()} bloquera sauf si l'option WNOHANG est donnée.

\begin{tcolorbox}[title={À retenir}]
Un processus parent peut créer un processus fils, qui hérite de la plupart des attributs du parent. Les processus zombies doivent être nettoyés par le parent à l'aide de l'appel système \texttt{wait*}.
\end{tcolorbox}

\newpage

\section{\texttt{fork()} - Création d'un processus fils}

L'appel \texttt{fork()} initialise un nouveau PCB basé sur la valeur du parent, et ce PCB est ajouté à la file d'attente des processus exécutables. À ce stade, il y a deux processus au même point d'exécution.

L'espace d'adressage du fils est une copie complète de celui du parent, avec une différence notable. L'appel \texttt{fork()} retourne deux fois : une fois au parent avec un PID supérieur à zéro, et une fois au fils avec un PID égal à zéro.

\begin{tcolorbox}[title={Intuition}]
L'appel \texttt{fork()} permet de dupliquer un processus existant, créant ainsi un nouveau processus fils qui est une copie presque identique du parent.
\end{tcolorbox}

\begin{lstlisting}
int main(int argc, char *argv[])
{
    int pid = fork();
    if( pid==0 ) {
        // child
        printf("parent=%d son=%d\n", getppid(), getpid());
    }
    else if( pid > 0 ) {
        // parent
        printf("parent=%d son=%d\n", getpid(), pid);
    }
    else { // print string associated with errno
        perror("fork() failed");
    }
    return 0;
}
\end{lstlisting}

L'ordre d'impression dépend de l'ordonnancement du système. La variable globale \texttt{errno} contient le numéro d'erreur de la dernière appel système.

\begin{tcolorbox}[title={Fiche Récapitulative}]
L'appel \texttt{fork()} crée un processus fils en dupliquant le processus parent. Le fils hérite de l'espace d'adressage du parent, et l'appel retourne deux fois pour distinguer le parent du fils.
\end{tcolorbox}

\end{document}