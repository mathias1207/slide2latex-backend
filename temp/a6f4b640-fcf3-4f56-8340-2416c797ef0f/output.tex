% !TEX program = xelatex
\documentclass[12pt]{report}
\usepackage[utf8]{inputenc}
\usepackage[en]{babel}

\usepackage{tcolorbox}
\usepackage{fontawesome5}
\usepackage{listings}
\usepackage{algorithm2e}
\usepackage{amsmath,amsfonts,amssymb,graphicx,float}
\usepackage{xcolor}
\usepackage{parskip}

% Configuration des listings
\lstset{
    basicstyle=\ttfamily\small,
    breaklines=true,
    frame=single,
    numbers=left,
    numberstyle=\tiny,
    numbersep=5pt,
    showstringspaces=false,
    language=C,
    keepspaces=true,
    columns=flexible,
    literate={*}{\textasteriskcentered}1
}

% Configuration des tcolorbox
\tcbuselibrary{most}
\newtcolorbox{mybox}[1][]{
    colback=white,
    colframe=black,
    coltitle=black,
    title=#1,
    fonttitle=\bfseries
}

% Configuration des marges
\usepackage[left=2.5cm,right=2.5cm,top=2.5cm,bottom=2.5cm]{geometry}

\graphicspath{{./images/}}
\title{osj}
\author{}

\begin{document}
\maketitle
\tableofcontents
\newpage

\section{Introduction}
% Cours : osj
\section{Process Creation and Termination}In an operating system, one process, which we can refer to as the parent, can create another process, known as the child. This is done by allocating and initializing a new Process Control Block (PCB). \begin{tcolorbox}[  colback=blue!10,  colframe=blue,  title={\fontfamily{lmr}\selectfont \faComment\ Simple Explanation},  fonttitle=\bfseries,  fontupper=\fontfamily{lmr}\selectfont,  boxrule=1pt,  sharp corners,]Think of the PCB as a sort of ID card for the process. It contains all the information the operating system needs to manage the process.\end{tcolorbox}\subsection{Parent and Child Processes}In POSIX, a child process inherits most of the parent's attributes such as the User ID (UID), open files, current working directory (cwd), etc. However, if these inherited files are not needed, they should be closed. \subsection{Process Control Block Movement}While a process is executing, its PCB moves between different queues according to the state change graph. These queues can be runnable, sleep/wait for event i (where i can be any number).\subsection{Process Termination}After a process dies, either by calling exit() or being interrupted, it becomes a zombie. The parent process uses the wait* system call to clear the zombie from the system. \section{Parent-Child Process Interaction}The parent can either wait for its child to finish or run in parallel. The wait*() function will block unless WNOHANG is given in 'options'.\subsection{Homework}Run 'ps auxwww' in the shell to see the PPID, which is the parent's PID. Also, read 'man 2 wait' to understand more about the wait system call family: wait, waitpid, waitid, wait3, wait4.
\begin{lstlisting}[language=C]pid_t wait4(pid_t, int *wstatus, int options, struct rusage *rusage); \end{lstlisting}

\begin{tcolorbox}[  colback=blue!10,  colframe=blue,  title={\fontfamily{lmr}\selectfont \faComment\ Vulgarisation simple},  fonttitle=\bfseries,  fontupper=\fontfamily{lmr}\selectfont,  boxrule=1pt,  sharp corners,]
Imaginez que vous êtes en train de cuisiner et que vous décidez de faire une recette qui nécessite de préparer deux choses en même temps. Vous pourriez faire appel à un assistant pour vous aider. Dans ce cas, vous (le processus parent) donneriez à votre assistant (le processus enfant) une copie de la recette (l'espace d'adresse) et vous continueriez à travailler en parallèle. Si votre assistant rencontre un problème (fork() échoue), il vous le fait savoir (la variable 'errno' contient le numéro d'erreur).
\end{tcolorbox}
\section{Création d'un processus enfant avec fork()} 

La fonction fork() est utilisée pour créer un nouveau processus, appelé processus enfant, à partir du processus actuel, appelé processus parent. Cette fonction initialise un nouveau Bloc de Contrôle de Processus (PCB) basé sur la valeur du processus parent et l'ajoute à la file d'attente des processus prêts à être exécutés. 

\subsection{Fonctionnement de fork()} 

Lorsque fork() est appelé, il crée un nouvel espace d'adresse pour le processus enfant qui est une copie complète de l'espace d'adresse du processus parent, à une différence près : la valeur de retour de fork(). En effet, fork() retourne deux fois : une fois dans le processus parent avec une valeur supérieure à zéro (le PID du processus enfant) et une fois dans le processus enfant avec une valeur de zéro. 

\subsection{Gestion des erreurs avec fork()} 

Si fork() échoue, il retourne une valeur négative. Dans ce cas, la variable globale 'errno' contient le numéro d'erreur du dernier appel système. Cette variable peut être utilisée pour afficher un message d'erreur approprié avec la fonction perror().
\begin{lstlisting}[language=C]
int main(int argc, char *argv[])
{
  int pid = fork();
  if( pid==0 ) {
   //
   // child
      //
      printf("parent=%d son=%d\n",
             getppid(), getpid());
  }
  else if( pid > 0 ) {
      //
      // parent
      //
      printf("parent=%d son=%d\n",
             getpid(), pid);
  }
  else { // print string associated
         // with errno   
      perror("fork() failed");
  }
  return 0;
}
\end{lstlisting}


\end{document}