\documentclass[12pt]{article}
\usepackage[utf8]{inputenc}
\usepackage[T1]{fontenc}
\usepackage[french]{babel}

\usepackage{tcolorbox}
\usepackage{fontawesome5}
\usepackage{listings}
\usepackage{amsmath}
\usepackage{xcolor}
\usepackage{geometry}
\usepackage{textcomp}
\DeclareUnicodeCharacter{00A0}{~}
\DeclareUnicodeCharacter{200B}{}

% Configuration simplifiée pour éviter les erreurs avec \hss
\lstset{
    basicstyle=\ttfamily\small,
    breaklines=true,
    breakatwhitespace=false,
    keepspaces=true,
    numbers=left,
    numberstyle=\tiny\color{gray},
    showspaces=false,
    showstringspaces=false,
    showtabs=false,
    tabsize=2
}

% Configuration minimale de tcolorbox
\tcbuselibrary{most}

% Ajouter cette commande pour les échappements
\newcommand{\escapesym}[1]{\texttt{\textbackslash{}#1}}

% Configuration des marges
\geometry{margin=2.5cm}

\title{oslzp}
\author{}
\date{}

\begin{document}
\maketitle
\tableofcontents
\newpage

\section{Processus de création et terminaison}

Un processus, appelé le "parent", peut en créer un autre, le "fils". Lors de cette création, un nouveau PCB (Process Control Block) est alloué et initialisé. Par exemple, vous pouvez exécuter la commande \texttt{ps auxwww} dans le shell pour voir que le PPID est l'identifiant du processus parent.

Dans le système POSIX, le processus fils hérite de la plupart des attributs du parent, tels que l'UID, les fichiers ouverts (qui devraient être fermés s'ils ne sont pas nécessaires) et le répertoire de travail courant.

Pendant son exécution, le PCB se déplace entre différentes files d'attente, selon le graphe de changement d'état. Les files d'attente incluent celles pour les processus exécutables, en sommeil ou en attente d'un événement spécifique.

Lorsqu'un processus meurt (par \texttt{exit()} ou interruption), il devient un zombie. Le parent utilise l'appel système \texttt{wait*} pour nettoyer le zombie du système. La famille d'appels système \texttt{wait} inclut \texttt{wait}, \texttt{waitpid}, \texttt{wait3}, et \texttt{wait4}. Par exemple :

\begin{lstlisting}
pid_t wait4(pid_t, int *wstatus, int options, struct rusage *rusage);
\end{lstlisting}

Le parent peut dormir ou attendre que son enfant se termine ou s'exécute en parallèle. L'appel \texttt{wait*()} bloquera sauf si \texttt{WNOHANG} est donné dans les options. Pour approfondir, consultez \texttt{man 2 wait}.

\section{\texttt{fork()} - Création d'un processus fils}

La fonction \texttt{fork()} initialise un nouveau PCB basé sur la valeur du parent et ajoute le PCB à la file d'attente des processus exécutables. Ainsi, deux processus existent au même point d'exécution.

L'espace d'adressage du fils est une copie complète de celui du parent, avec une différence notable. La fonction \texttt{fork()} retourne deux fois : une fois au parent avec un \texttt{pid} supérieur à zéro, et une fois au fils avec un \texttt{pid} égal à zéro.

\begin{lstlisting}
int main(int argc, char *argv[])
{
    int pid = fork();
    if( pid==0 ) {
        // child
        printf("parent=%d son=%dn", getppid(), getpid());
    }
    else if( pid > 0 ) {
        // parent
        printf("parent=%d son=%dn", getpid(), pid);
    }
    else { // print string associated with errno
        perror("fork() failed");
    }
    return 0;
}
\end{lstlisting}

La variable globale \texttt{errno} contient le numéro d'erreur de la dernière appel système.

\begin{tcolorbox}[title={À retenir}]
La fonction \texttt{fork()} permet de créer un processus fils, qui hérite de l'espace d'adressage du parent. Le processus fils et le parent peuvent s'exécuter en parallèle, et le nettoyage des processus zombies est essentiel pour libérer les ressources système.
\end{tcolorbox}

\end{document}