\documentclass[12pt]{article}
\usepackage[utf8]{inputenc}
\usepackage[T1]{fontenc}
\usepackage{geometry}
\usepackage{tcolorbox}
\usepackage{fontawesome5}
\usepackage{listings}
\usepackage{amsmath}
\usepackage{xcolor}
\usepackage{textcomp}
\usepackage{graphicx}
\usepackage{amsfonts}
\usepackage{amssymb}
\usepackage[version=4]{mhchem}
\usepackage{hyperref}
\usepackage{stmaryrd}
\graphicspath{ {./images/} }
\usepackage{float}
\usepackage[linesnumbered,ruled,vlined]{algorithm2e}
\usepackage{fontawesome}
\usepackage{amsthm}
\usepackage{pifont}
\DeclareUnicodeCharacter{00A0}{~}
\DeclareUnicodeCharacter{200B}{}
\usepackage[french]{babel}

% Configuration simplifiée pour éviter les erreurs avec \hss
\lstset{
    basicstyle=\ttfamily\small,
    breaklines=true,
    breakatwhitespace=false,
    keepspaces=true,
    numbers=left,
    numberstyle=\tiny\color{gray},
    showspaces=false,
    showstringspaces=false,
    showtabs=false,
    tabsize=2
}

% Configuration minimale de tcolorbox
\tcbuselibrary{most}

% Ajouter cette commande pour les échappements
\newcommand{\escapesym}[1]{\texttt{\textbackslash{}#1}}

% Configuration des marges
\geometry{margin=2.5cm}

\title{fmf}
\author{}
\date{}

\begin{document}
\maketitle
\tableofcontents
\newpage

\section{Fourier Family}

Consider the Fourier family of orthonormal functions:

\[
\{\ldots, \beta_{-2}^F(t), \beta_{-1}^F(t), \beta_0^F(t), \beta_1^F(t), \beta_2^F(t), \ldots\}
\]

where for \( k \in \mathbb{Z} \):

\[
\beta_k^F(t) := \exp(i2\pi kt) \quad \text{for } t \in [0,1].
\]

Recall that \( i^2 := -1 \) and that 

\[
\exp(i2\pi kt) = e^{i2\pi kt} = \cos(2\pi kt) + i \sin(2\pi kt).
\]

Are the functions orthonormal?

\subsection{Orthonormality Check}

Note that the inner-product is defined as:

\[
\langle \beta_k^F, \beta_l^F \rangle = \int_0^1 \beta_k^{F*}(t) \beta_l^F(t) \, dt
\]

When \( k = l \):

\[
\langle \beta_k^F, \beta_l^F \rangle = \int_0^1 \beta_k^{F*}(t) \beta_k^F(t) \, dt = \int_0^1 e^{-i2\pi kt} e^{i2\pi kt} \, dt = \int_0^1 \exp(0) \, dt = 1
\]

When \( k \neq l \):

\[
\langle \beta_k^F, \beta_l^F \rangle = \int_0^1 e^{-i2\pi kt} e^{i2\pi lt} \, dt = \int_0^1 e^{i2\pi (l-k)t} \, dt
\]

\[
= \int_0^1 \cos(2\pi (l-k)t) + i \sin(2\pi (l-k)t) \, dt = 0
\]

\begin{tcolorbox}[title={À retenir}]
Les fonctions de la famille de Fourier sont orthonormales si et seulement si:

\[
\langle \beta_k^F(t), \beta_l^F(t) \rangle = \delta_{k,l} = 
\begin{cases} 
1, & k = l \\ 
0, & \text{sinon.}
\end{cases}
\]
\end{tcolorbox}

\newpage

\section{Signal Representation}

Consider a function \( \phi : [0,1) \rightarrow \mathbb{R} \).

Its optimal representation using the Fourier family is given by:

\[
\hat{\phi}^F(t) = \sum_{k=-N}^{N} \tilde{\phi}_k^F \beta_k^F(t)
\]

The corresponding optimal coefficients (in the sense of minimal MSE) are:

\[
\tilde{\phi}_k^F = \langle \beta_k^F, \phi \rangle = \int_0^1 \phi(t) e^{-i2\pi kt} \, dt
\]

The error term is defined by:

\[
\mathcal{E}(t) = \phi(t) - \hat{\phi}^F(t)
\]

The MSE is defined as:

\[
\|\mathcal{E}(t)\|^2 = \langle \mathcal{E}(t), \mathcal{E}(t) \rangle = \int_0^1 \left( \phi(t) - \hat{\phi}^F(t) \right) \left( \phi(t) - \hat{\phi}^F(t) \right)^* \, dt
\]

The minimal MSE obtained using the optimal coefficients is (prove!):

\[
\|\mathcal{E}(t)\|^2 = \int_0^1 \phi^2(t) \, dt - \sum_{k=-N}^{N} |\tilde{\phi}_k^F|^2
\]

\begin{tcolorbox}[colback=red!5!white, colframe=red!75!black, title={\faBookmark\hspace{0.5em}Fiche Récapitulative}]
  
\textbf{Objectif :} Résumer la représentation optimale d'un signal.

\textbf{Principe central :} Utiliser la famille de Fourier pour représenter un signal de manière optimale.

\vspace{0.4em}
\textbf{Points essentiels :}  
- Représentation optimale avec la famille de Fourier
- Coefficients optimaux pour minimiser l'erreur quadratique moyenne
- Définition de l'erreur et du MSE

\vspace{0.4em}
\textbf{Formules clés :}  
\begin{itemize}
\item \(\hat{\phi}^F(t) = \sum_{k=-N}^{N} \tilde{\phi}_k^F \beta_k^F(t)\)
\item \(\tilde{\phi}_k^F = \int_0^1 \phi(t) e^{-i2\pi kt} \, dt\)
\end{itemize}
  
\end{tcolorbox}

\newpage

\section{Conjugate-Symmetry of Representation}

Note that we consider a function \( \phi : [0,1) \rightarrow \mathbb{R} \).

The \( k \)-th coefficient:

\[
\tilde{\phi}_k^F = \langle \beta_k^F, \phi \rangle = \int_0^1 \phi(t) e^{-i2\pi kt} \, dt
\]

The \((-k)\)-th coefficient:

\[
\tilde{\phi}_{-k}^F = \langle \beta_{-k}^F, \phi \rangle = \int_0^1 \phi(t) e^{i2\pi kt} \, dt
\]

\[
\tilde{\phi}_k^F = (\tilde{\phi}_{-k}^F)^*
\]

\newpage

\section{Fourier Family}

In general, the Fourier family of orthonormal functions is:

\[
\{\ldots, \alpha_{-2}^F(t), \alpha_{-1}^F(t), \alpha_0^F(t), \alpha_1^F(t), \alpha_2^F(t), \ldots\}
\]

where for \( k \in \mathbb{Z} \):

\[
\alpha_k^F(t) := \exp\left(\frac{i2\pi kt}{T}\right) \quad \text{for } t \in [0,T].
\]

Its optimal representation using the Fourier family is given by:

\[
\hat{\phi}^F(t) = \sum_{k=-N}^{N} \tilde{\phi}_k^F \alpha_k^F(t)
\]

\[
\tilde{\phi}_k^F = \frac{1}{T} \int_0^T \phi(t) e^{-\frac{i2\pi kt}{T}} \, dt
\]

\newpage

\section{Example}

Consider Fourier family of orthonormal functions:

\[
\{\ldots, \alpha_{-2}^F(t), \alpha_{-1}^F(t), \alpha_0^F(t), \alpha_1^F(t), \alpha_2^F(t), \ldots\}
\]

where for \( k \in \mathbb{Z} \):

\[
\alpha_k^F(t) := \exp\left(\frac{i2\pi kt}{2}\right) \quad \text{for } t \in [0,2].
\]

Consider a real-valued signal \( \phi(t) = t^2 \) defined for \( t = [0,2) \).

Its optimal representation using the Fourier family is given by:

\[
\hat{\phi}^F(t) = \sum_{k=-\infty}^{\infty} \tilde{\phi}_k^F \alpha_k^F(t)
\]

Assume the error signal is zero:

\[
\phi(t) = t^2 = \hat{\phi}^F(t) = \sum_{k=-\infty}^{\infty} \tilde{\phi}_k^F \alpha_k^F(t) \quad \text{for } t \in [0,2)
\]

What is the optimal coefficient \( \tilde{\phi}_0^F \)?

Show that the optimal coefficient:

\[
\tilde{\phi}_k^F = \frac{2(1 + i\pi k)}{\pi^2 k^2} \quad \text{for } k \in \mathbb{Z} \text{ and } k \neq 0.
\]

Show that:

\[
\sum_{k=1}^{\infty} \frac{(-1)^{k+1}}{k^2} = \frac{\pi^2}{12}
\]

\newpage

\subsection{Optimal Coefficient Calculation}

What is the optimal coefficient \( \tilde{\phi}_0^F \)?

\[
\tilde{\phi}_0^F = \frac{1}{2} \int_0^2 \phi(t) e^{-\frac{i2\pi 0 t}{2}} \, dt
\]

\[
= \frac{1}{2} \int_0^2 \phi(t) \, dt = \frac{1}{2} \int_0^2 t^2 \, dt = \frac{4}{3}
\]

\subsection{Show the Optimal Coefficient}

Show that the optimal coefficient:

\[
\tilde{\phi}_k^F = \frac{2(1 + i\pi k)}{\pi^2 k^2} \quad \text{for } k \in \mathbb{Z} \text{ and } k \neq 0.
\]

\[
\tilde{\phi}_k^F = \frac{1}{2} \int_0^2 \phi(t) e^{-\frac{i2\pi kt}{2}} \, dt = \frac{1}{2} \int_0^2 t^2 e^{-i\pi kt} \, dt
\]

\[
= \frac{1}{2} \left( \left[ \frac{t^2 e^{-i\pi kt}}{-i\pi k} \right]_0^2 - \int_0^2 \frac{2}{-i\pi k} t e^{-i\pi kt} \, dt \right)
\]

\[
= \frac{1}{2} \left( \frac{4}{-i\pi k} + \frac{2}{i\pi k} \frac{-2}{i\pi k} \right) = \frac{2(1 + i\pi k)}{\pi^2 k^2}
\]

\newpage

\subsection{Show the Series Result}

Show that:

\[
\sum_{k=1}^{\infty} \frac{(-1)^{k+1}}{k^2} = \frac{\pi^2}{12}
\]

Recall that the error signal is zero:

\[
\phi(t) = t^2 = \hat{\phi}^F(t) = \sum_{k=-\infty}^{\infty} \tilde{\phi}_k^F \alpha_k^F(t) \quad \text{for } t \in [0,2)
\]

Observe when \( t = 1 \):

\[
1 = \frac{4}{3} + \sum_{k=1}^{\infty} \frac{2(1 + i\pi k)}{\pi^2 k^2} e^{i\pi k} + \sum_{k=-\infty}^{-1} \frac{2(1 + i\pi k)}{\pi^2 k^2} e^{i\pi k}
\]

\[
= \frac{4}{3} + \sum_{k=1}^{\infty} \frac{2(1 + i\pi k)}{\pi^2 k^2} (-1)^k + \sum_{k=1}^{\infty} \frac{2(1 - i\pi k)}{\pi^2 k^2} (-1)^k
\]

\[
= \frac{4}{3} + \sum_{k=1}^{\infty} \frac{4}{\pi^2 k^2} (-1)^k
\]

\[
\sum_{k=1}^{\infty} \frac{4}{\pi^2 k^2} (-1)^k = -\frac{1}{3}
\]

\[
\sum_{k=1}^{\infty} \frac{4}{\pi^2 k^2} (-1)^{k+1} = \frac{1}{3}
\]

\[
\sum_{k=1}^{\infty} \frac{(-1)^{k+1}}{k^2} = \frac{\pi^2}{12}
\]

\newpage

\section{Fourier Transformation}

\[
\tilde{\phi}_k^F = \frac{1}{T} \int_0^T \phi(t) e^{-\frac{i2\pi kt}{T}} \, dt
\]

\begin{tabular}{|c|c|}
\hline
Signal & Fourier representation \\
\hline
\(\phi(t)\) & \(\tilde{\phi}_k^F\) \\
\hline
\(f(t) = \phi(t-a)\) & \(\tilde{f}_k^F = ?\) \\
\hline
\(g(t) = \phi(at)\) & \(\tilde{g}_k^F = ?\) \\
\hline
\(h(t) = \phi(t)\exp(iat)\) & \(\tilde{h}_k^F = ?\) \\
\hline
\end{tabular}

\end{document}