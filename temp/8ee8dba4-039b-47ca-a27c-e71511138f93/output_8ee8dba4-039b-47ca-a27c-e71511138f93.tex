\documentclass[12pt]{article}
\usepackage[utf8]{inputenc}
\usepackage[T1]{fontenc}
\usepackage[french]{babel}

\usepackage{tcolorbox}
\usepackage{fontawesome5}
\usepackage{listings}
\usepackage{amsmath}
\usepackage{xcolor}
\usepackage{geometry}
\usepackage{textcomp}
\DeclareUnicodeCharacter{00A0}{~}
\DeclareUnicodeCharacter{200B}{}

% Configuration simplifiée pour éviter les erreurs avec \hss
\lstset{
    basicstyle=\ttfamily\small,
    breaklines=true,
    breakatwhitespace=false,
    keepspaces=true,
    numbers=left,
    numberstyle=\tiny\color{gray},
    showspaces=false,
    showstringspaces=false,
    showtabs=false,
    tabsize=2
}

% Configuration minimale de tcolorbox
\tcbuselibrary{most}

% Ajouter cette commande pour les échappements
\newcommand{\escapesym}[1]{\texttt{\textbackslash{}#1}}

% Configuration des marges
\geometry{margin=2.5cm}

\title{oslzp}
\author{}
\date{}

\begin{document}
\maketitle
\tableofcontents
\newpage

\section{Processus de création et de terminaison}

Un processus, appelé le "parent", peut en créer un autre, appelé "enfant". Lors de cette création, un nouveau PCB (Process Control Block) est alloué et initialisé. Pour explorer cela, vous pouvez exécuter la commande \texttt{ps auxwww} dans le shell, où le PPID représente le PID du parent.

Dans le système POSIX, le processus enfant hérite de la plupart des attributs du parent, tels que l'UID, les fichiers ouverts (qui devraient être fermés si inutiles), le répertoire de travail courant, etc.

Pendant son exécution, le PCB se déplace entre différentes files d'attente, selon le graphe de changement d'état. Les files d'attente incluent celles pour les processus exécutables, en sommeil ou en attente d'un événement.

Lorsqu'un processus meurt (par \texttt{exit()} ou interruption), il devient un zombie. Le parent utilise l'appel système \texttt{wait*} pour supprimer le zombie du système. La famille d'appels système \texttt{wait} comprend \texttt{wait}, \texttt{waitpid}, \texttt{waitid}, \texttt{wait3}, et \texttt{wait4}. Par exemple :

\begin{lstlisting}
pid_t wait4(pid_t, int *wstatus, int options, struct rusage *rusage);
\end{lstlisting}

Le parent peut dormir ou attendre que son enfant se termine ou s'exécute en parallèle. L'appel \texttt{wait*()} bloquera à moins que l'option \texttt{WNOHANG} ne soit spécifiée. Pour approfondir, lisez \texttt{man 2 wait}.

\section{fork() - Création d'un processus enfant}

La fonction \texttt{fork()} initialise un nouveau PCB basé sur la valeur du parent et ajoute le PCB à la file d'attente des processus exécutables. Ainsi, deux processus existent au même point d'exécution.

L'espace d'adressage du nouvel enfant est une copie complète de celui du parent, avec une différence notable. La fonction \texttt{fork()} retourne deux fois : une fois au parent avec un \texttt{pid > 0}, et une fois à l'enfant avec \texttt{pid = 0}.

\begin{tcolorbox}[title={Intuition}]
L'ordre d'impression peut être déroutant car les deux processus s'exécutent simultanément. La variable globale \texttt{errno} contient le numéro d'erreur de la dernière appel système.
\end{tcolorbox}

Voici un exemple de code illustrant l'utilisation de \texttt{fork()} :

\begin{lstlisting}
int main(int argc, char *argv[])
{
    int pid = fork();
    if( pid == 0 ) {
        // Enfant
        printf("parent=%d son=%dn", getppid(), getpid());
    }
    else if( pid > 0 ) {
        // Parent
        printf("parent=%d son=%dn", getpid(), pid);
    }
    else { // Affiche la chaine associee a errno
        perror("fork() failed");
    }
    return 0;
}
\end{lstlisting}

\end{document}