\documentclass[12pt]{article}
\usepackage[utf8]{inputenc}
\usepackage[french]{babel}

\usepackage{tcolorbox}
\usepackage{fontawesome5}
\usepackage{listings}
\usepackage{amsmath}
\usepackage{xcolor}
\usepackage{geometry}

\% Configuration minimale de tcolorbox
\tcbuselibrary{most}

\% Configuration minimale des listings
\lstset{breaklines=true, basicstyle=\ttfamily\small}

\% Configuration des marges
\geometry{margin=2.5cm}

\title{osj}
\author{}
\date{}

\begin{document}
\maketitle
\tableofcontents
\newpage

\% Cours : osj
Process creation \& termination
• One process (the “parent”) can create another (the “child”)
– A new PCB is allocated and initialized
– Homework: run ‘ps auxwww’ in the shell; PPID is the parent’s PID
• In POSIX, child process inherits most of parent’s attributes
– UID, open files (should be closed if unneeded; why?), cwd, etc.
• While executing, PCB moves between different queues
– According to state change graph 
– Queues: runnable, sleep/wait for event i (i=1,2,3…)
• After a process dies (exit()s / interrupted), it becomes a zombie
– Parent uses wait* syscall to clear zombie from the system (why?)
– Wait syscall family: wait, waitpid, waitid, wait3, wait4; example:
– pid\_t wait4(pid\_t, int *wstatus, int options, struct rusage *rusage); 
• Parent can sleep/wait for its child to finish or run in parallel
– wait*() will block unless WNOHANG given in ‘options’
– Homework: read ‘man 2 wait’
OS (234123) - processes \& signals
7

int main(int argc, char *argv[])
{
  int pid = fork();
  if( pid==0 ) { 
   //
   // child
      //
      printf(“parent=\%d son=\%d\n”,
             getppid(), getpid());
  }
  else if( pid > 0 ) {
      //
      // parent
      //
      printf(“parent=\%d son=\%d\n”,
             getpid(), pid);
  }
  else { // print string associated
         // with errno   
      perror(“fork() failed”); 
  }
  return 0;
}
• fork() initializes a new PCB
– Based on parent’s value
– PCB added to runnable queue
• Now there are 2 processes
– At same execution point
• Child’s new address space 
– Complete copy of parent’s 
space, with one difference…
• fork() returns twice
– At the parent, with pid>0
– At the child, with pid=0
• What’s the printing order?
• ‘errno’ – a global variable
– Holds error num of last syscall
OS (234123) - processes \& signals
8
fork() – spawn a child process


\section*{Fiche Récapitulative}

Génère une fiche récapitulative pour le chapitre suivant :
Process creation \& termination

Contenu du chapitre :
Process creation \& termination
• One process (the “parent”) can create another (the “child”)
– A new PCB is allocated and initialized
– Homework: run ‘ps auxwww’ in the shell; PPID is the parent’s PID
• In POSIX, child process inherits most of parent’s attributes
– UID, open files (should be closed if unneeded; why?), cwd, etc.
• While executing, PCB moves between different queues
– According to state change graph 
– Queues: runnable, sleep/wait for event i (i=1,2,3…)
• After a process dies (exit()s / interrupted), it becomes a zombie
– Parent uses wait* syscall to clear zombie from the system (why?)
– Wait syscall family: wait, waitpid, waitid, wait3, wait4; example:
– pid\_t wait4(pid\_t, int *wstatus, int options, struct rusage *rusage); 
• Parent can sleep/wait for its child to finish or run in parallel
– wait*() will block unless WNOHANG given in ‘options’
– Homework: read ‘man 2 wait’
OS (234123) - processes \& signals
7 int main(int argc, char *argv[])
{
  int pid = fork();
  if( pid==0 ) { 
   //
   // child
      //
      printf(“parent=\%d son=\%d\n”,
             getppid(), getpid());
  }
  else if( pid > 0 ) {
      //
      // parent
      //
      printf(“parent=\%d son=\%d\n”,
             getpid(), pid);
  }
  else { // print string associated
         // with errno   
      perror(“fork() failed”); 
  }
  return 0;
}
• fork() initializes a new PCB
– Based on parent’s value
– PCB added to runnable queue
• Now there are 2 processes
– At same execution point
• Child’s new address space 
– Complete copy of parent’s 
space, with one difference…
• fork() returns twice
– At the parent, with pid>0
– At the child, with pid=0
• What’s the printing order?
• ‘errno’ – a global variable
– Holds error num of last syscall
OS (234123) - processes \& signals
8
fork() – spawn a child process

La fiche doit inclure :
1. Les points clés
2. Les formules importantes
3. Les concepts essentiels
4. Les applications pratiques

Format : Utilise le même format que les autres blocs avec tcolorbox.


\end{document}