\documentclass[12pt]{article}
\usepackage[utf8]{inputenc}
\usepackage[T1]{fontenc}
\usepackage[french]{babel}

\usepackage{tcolorbox}
\usepackage{fontawesome5}
\usepackage{listings}
\usepackage{amsmath}
\usepackage{xcolor}
\usepackage{geometry}
\usepackage{textcomp}
\DeclareUnicodeCharacter{00A0}{~}
\DeclareUnicodeCharacter{200B}{}

% Configuration simplifiée pour éviter les erreurs avec \hss
\lstset{
    basicstyle=\ttfamily\small,
    breaklines=true,
    breakatwhitespace=false,
    keepspaces=true,
    numbers=left,
    numberstyle=\tiny\color{gray},
    showspaces=false,
    showstringspaces=false,
    showtabs=false,
    tabsize=2
}

% Configuration minimale de tcolorbox
\tcbuselibrary{most}

% Ajouter cette commande pour les échappements
\newcommand{\escapesym}[1]{\texttt{\textbackslash{}#1}}

% Configuration des marges
\geometry{margin=2.5cm}

\title{osjlm}
\author{}
\date{}

\begin{document}
\maketitle
\tableofcontents
\newpage

section{Création et terminaison de processus}

Un processus, appelé le "parent", peut en créer un autre, le "fils". Lors de cette création, un nouveau PCB (Process Control Block) est alloué et initialisé. Pour observer cela, vous pouvez exécuter la commande \texttt{ps auxwww} dans le shell, où le PPID représente le PID du parent.

En POSIX, le processus fils hérite de la plupart des attributs du parent, tels que l'UID, les fichiers ouverts (qui devraient être fermés s'ils ne sont pas nécessaires), le répertoire de travail courant, etc.

Pendant son exécution, le PCB se déplace entre différentes files d'attente selon le graphe de changement d'état. Ces files incluent les processus exécutables, en sommeil ou en attente d'un événement.

Lorsqu'un processus meurt (par \texttt{exit()} ou interruption), il devient un zombie. Le parent utilise l'appel système \texttt{wait*} pour nettoyer le zombie du système. La famille d'appels système \texttt{wait} inclut \texttt{wait}, \texttt{waitpid}, \texttt{waitid}, \texttt{wait3}, et \texttt{wait4}. Par exemple :

\begin{lstlisting}
pid_t wait4(pid_t, int *wstatus, int options, struct rusage *rusage);
\end{lstlisting}

Le parent peut dormir ou attendre que son enfant se termine ou s'exécute en parallèle. L'appel \texttt{wait*()} bloquera à moins que \texttt{WNOHANG} ne soit donné dans les options. Pour approfondir, consultez \texttt{man 2 wait}.

\begin{tcolorbox}[title={À retenir}]
Un processus parent peut créer un processus fils qui hérite de ses attributs. Lorsqu'un processus se termine, il devient un zombie jusqu'à ce que le parent le nettoie avec \texttt{wait*}.
\end{tcolorbox}

section{fork() - Création d'un processus fils}

La fonction \texttt{fork()} initialise un nouveau PCB basé sur la valeur du parent et ajoute le PCB à la file d'attente des processus exécutables. Ainsi, deux processus existent au même point d'exécution.

L'espace d'adressage du fils est une copie complète de celui du parent, à une différence près. La fonction \texttt{fork()} retourne deux fois : une fois au parent avec un \texttt{pid} supérieur à zéro, et une fois au fils avec un \texttt{pid} égal à zéro.

\begin{lstlisting}
int main(int argc, char *argv[])
{
    int pid = fork();
    if( pid==0 ) {
        // child
        printf("parent=%d son=%dn", getppid(), getpid());
    }
    else if( pid > 0 ) {
        // parent
        printf("parent=%d son=%dn", getpid(), pid);
    }
    else { // print string associated with errno
        perror("fork() failed");
    }
    return 0;
}
\end{lstlisting}

L'ordre d'impression dépend de l'exécution des processus. La variable globale \texttt{errno} contient le numéro d'erreur de la dernière appel système.

\begin{tcolorbox}[title={Fiche Récapitulative}]
\begin{itemize}
    \item \texttt{fork()} crée un processus fils avec un nouvel espace d'adressage.
    \item \texttt{fork()} retourne deux fois, une fois au parent et une fois au fils.
    \item \texttt{errno} est utilisé pour capturer les erreurs des appels système.
\end{itemize}
\end{tcolorbox}

\end{document}