\documentclass[12pt]{article}
\usepackage[utf8]{inputenc}
\usepackage[T1]{fontenc}
\usepackage[french]{babel}

\usepackage{tcolorbox}
\usepackage{fontawesome5}
\usepackage{listings}
\usepackage{amsmath}
\usepackage{xcolor}
\usepackage{geometry}
\usepackage{textcomp}
\DeclareUnicodeCharacter{00A0}{~}
\DeclareUnicodeCharacter{200B}{}

% Configuration simplifiée pour éviter les erreurs avec \hss
\lstset{
    basicstyle=\ttfamily\small,
    breaklines=true,
    breakatwhitespace=false,
    keepspaces=true,
    numbers=left,
    numberstyle=\tiny\color{gray},
    showspaces=false,
    showstringspaces=false,
    showtabs=false,
    tabsize=2
}

% Configuration minimale de tcolorbox
\tcbuselibrary{most}

% Ajouter cette commande pour les échappements
\newcommand{\escapesym}[1]{\texttt{\textbackslash{}#1}}

% Configuration des marges
\geometry{margin=2.5cm}

\title{fourier}
\author{}
\date{}

\begin{document}
\maketitle
\tableofcontents
\newpage

\section{Introduction}

\subsection{Signals and Systems in Continuous Time}

\begin{center}
% FIGURE: Diagram of a system with input $x(t)$ and output $y(t)$
\end{center}

The input signal $x(t)$ is defined for $t \in (-\infty, \infty)$ and is a periodic extension of a signal defined over a limited interval with period $T$. The output signal $y(t) = \mathcal{H} \circ x(t)$ is also defined for $t \in (-\infty, \infty)$.

\subsection{Properties of Systems}

The system $\mathcal{H}$ is said to be \textit{linear} if:

\begin{equation}
\mathcal{H}(k_1 x_1(t) + k_2 x_2(t)) = k_1 \mathcal{H}(x_1(t)) + k_2 \mathcal{H}(x_2(t)) = k_1 y_1(t) + k_2 y_2(t)
\end{equation}

The system $\mathcal{H}$ is said to be \textit{shift-invariant} if:

\begin{equation}
\mathcal{H}(\mathcal{T}_{t_0}(x(t))) = \mathcal{T}_{t_0}(y(t)) \quad \text{where} \quad \mathcal{T}_{t_0}(x(t)) = x(t - t_0)
\end{equation}

\subsection{Examples}

\begin{center}
\begin{tabular}{|c|c|c|}
\hline
System & Linear & Shift-Invariant \\
\hline
$y(t) = x^2(t)$ &  &  \\
$y(t) = x(t) + a$ &  &  \\
$y(t) = ax(t)$ & \checkmark & \checkmark \\
Optimal sampling &  &  \\
\hline
\end{tabular}
\end{center}

\section{Linear Shift-Invariant Systems}

\subsection{Input Signal and System}

The input signal $x(t)$ is defined for $t \in (-\infty, \infty)$ and is a periodic extension of a signal defined over a limited interval with period 1.

\begin{center}
% FIGURE: Diagram of a system with input $x(t)$ and output $y(t)$
\end{center}

The system $\mathcal{H}$ has an impulse response $\tilde{h}(t)$ with a basic period such that $\tilde{h} : [0,1) \rightarrow \mathbb{R}$. This impulse response is periodically extended into $h(t)$.

\section{Convolution Systems}

Consider an input signal $x(t)$ defined for $t \in [0,1]$. The convolution system is given by:

\begin{equation}
y(t) = \int_0^1 x(\tau) h(t - \tau) d\tau
\end{equation}

We use here the orthonormal family of Fourier functions, defined over $[0,1)$ for $k \in \mathbb{Z}$:

\begin{equation}
\beta_k^F(t) = \exp(i2\pi kt)
\end{equation}

\section{Fourier-based Analysis of Systems}

Using the Fourier family, the representations are:

\begin{align}
x(t) &= \sum_{k=-\infty}^{\infty} a_k \beta_k^F(t) \\
a_k &= \langle \beta_k^F, x \rangle = \int_0^1 x(t) \exp(-i2\pi kt) dt
\end{align}

\begin{align}
h(t) &= \sum_{k=-\infty}^{\infty} b_k \beta_k^F(t) \\
b_k &= \langle \beta_k^F, h \rangle = \int_0^1 h(t) \exp(-i2\pi kt) dt
\end{align}

\subsection{Convolution Systems}

The convolution system is expressed as:

\begin{align}
y(t) &= \int_0^1 x(\tau) h(t - \tau) d\tau \\
&= \sum_{k=-\infty}^{\infty} a_k b_k \beta_k^F(t)
\end{align}

\begin{equation}
c_k = a_k b_k
\end{equation}

\section{Derivation System}

Consider an input signal $x(t)$ defined for $t \in [0,1]$. The derivation system is:

\begin{equation}
y(t) = \frac{d}{dt} x(t)
\end{equation}

with boundary values obeying the cyclic continuity: $x(0) = x(1)$.

\begin{equation}
c_k = i2\pi k a_k
\end{equation}

\section{DFT}

The Discrete Fourier Transform (DFT) matrix is given by:

\begin{equation}
[\text{DFT}] = \frac{1}{\sqrt{N}}
\begin{bmatrix}
W^{*0 \cdot 0} & W^{*1 \cdot 0} & \cdots & W^{*(N-1) \cdot 0} \\
W^{*0 \cdot 1} & W^{*1 \cdot 1} & \cdots & W^{*(N-1) \cdot 1} \\
\vdots & \vdots & \ddots & \vdots \\
W^{*0 \cdot (N-1)} & W^{*1 \cdot (N-1)} & \cdots & W^{*(N-1) \cdot (N-1)}
\end{bmatrix}
\end{equation}

where $W = \exp\left(\frac{i2\pi}{N}\right)$. Note that the DFT matrix is symmetric and unitary.

\section{Representation of a Discrete Signal in the DFT Domain}

The representation of the discrete signal $\phi \in \mathbb{R}^N$ is:

\begin{equation}
\phi^F = [\text{DFT}] \phi
\end{equation}

Observe that:

\begin{align}
[\text{DFT}]^* \phi^F &= [\text{DFT}]^* [\text{DFT}] \phi \\
[\text{DFT}]^* \phi^F &= \phi
\end{align}

Above is the inverse DFT procedure.

\section{Example}

Consider the following discrete signal of $N$ samples:

For $n = 0, 1, 2, \ldots, (N-1)$, $\phi(n) = \cos\left(\frac{2\pi k_0}{N} n\right)$ where $k_0 \in \{0, 1, 2, \ldots, (N-1)\}$.

The $k^{th}$ component of the DFT-domain representation of the above signal is:

\begin{align}
\phi^F(k) &= \frac{1}{\sqrt{N}} \sum_{n=0}^{N-1} W^{*k \cdot n} \phi(n) \\
&= \frac{1}{\sqrt{N}} \left(\frac{1}{2} \sum_{n=0}^{N-1} W^{*k \cdot n} (W^{k_0 \cdot n} + W^{-k_0 \cdot n})\right)
\end{align}

When $k = k_0$:

\begin{equation}
\sum_{n=0}^{N-1} W^{*k \cdot n} (W^{k_0 \cdot n}) = N
\end{equation}

When $k \neq k_0$:

\begin{equation}
\sum_{n=0}^{N-1} W^{*k \cdot n} (W^{k_0 \cdot n}) = \frac{(W^{-(k-k_0)})^N - 1}{W^{-(k-k_0)} - 1} = 0
\end{equation}

Similarly:

\begin{equation}
\phi^F(k) = \frac{1}{\sqrt{N}} \left(\frac{\sqrt{N}}{2} (\delta_{k,k_0} + \delta_{k,-k_0})\right)
\end{equation}

For example, for $N = 9$ and $k_0 = 3$, the vector of DFT coefficients is:

\begin{equation}
\left[0, 0, 0, \frac{3}{2}, 0, 0, \frac{3}{2}, 0, 0\right]^\top
\end{equation}

\end{document}