documentclass[12pt]{article}
usepackage[utf8]{inputenc}
usepackage[T1]{fontenc}
usepackage[french]{babel}
usepackage{tcolorbox}
usepackage{fontawesome5}
usepackage{listings}
usepackage{amsmath}
usepackage{xcolor}
usepackage{geometry}
usepackage{textcomp}
\DeclareUnicodeCharacter{00A0}{~}
\DeclareUnicodeCharacter{200B}{}

% Configuration simplifiée pour éviter les erreurs avec \hss
\lstset{
    basicstyle=ttfamilysmall,
    breaklines=true,
    breakatwhitespace=false,
    keepspaces=true,
    numbers=left,
    numberstyle=tinycolor{gray},
    showspaces=false,
    showstringspaces=false,
    showtabs=false,
    tabsize=2
}

% Configuration minimale de tcolorbox
tcbuselibrary{most}

% Configuration des marges
\geometry{margin=2.5cm}

title{Document LaTeX}
author{}
date{}

begin{document}
\maketitle
tableofcontents
newpage

documentclass[12pt]{article}
usepackage[utf8]{inputenc}
usepackage[T1]{fontenc}
usepackage[french]{babel}

usepackage{tcolorbox}
usepackage{fontawesome5}
usepackage{listings}
usepackage{amsmath}
usepackage{xcolor}
usepackage{geometry}
usepackage{textcomp}
\DeclareUnicodeCharacter{00A0}{~}
\DeclareUnicodeCharacter{200B}{}

% Configuration simplifiée pour éviter les erreurs avec \hss
\lstset{
    basicstyle=ttfamilysmall,
    breaklines=true,
    breakatwhitespace=false,
    keepspaces=true,
    numbers=left,
    numberstyle=tinycolor{gray},
    showspaces=false,
    showstringspaces=false,
    showtabs=false,
    tabsize=2
}

% Configuration minimale de tcolorbox
tcbuselibrary{most}

% Ajouter cette commande pour les échappements
newcommand{escapesym}[1]{texttt{textbackslash{}#1}}

% Configuration des marges
\geometry{margin=2.5cm}

title{osjk}
author{}
date{}

begin{document}
\maketitle
tableofcontents
newpage

section{Processus de création et terminaison}

Un processus, appelé le "parent", peut en créer un autre, le "fils". Lors de cette création, un nouveau PCB (Process Control Block) est alloué et initialisé. Pour explorer cela, vous pouvez exécuter la commande texttt{ps auxwww} dans le shell, où le PPID représente le PID du parent.

En POSIX, le processus fils hérite de la plupart des attributs du parent, tels que l'UID, les fichiers ouverts (qui devraient être fermés si inutiles), le répertoire de travail courant, etc.

Pendant l'exécution, le PCB se déplace entre différentes files d'attente, selon le graphe de changement d'état. Les files d'attente incluent les états exécutable, en sommeil/attente pour un événement (i) (avec (i=1,2,3...)).

Lorsqu'un processus meurt (par texttt{exit( )} ou interruption), il devient un zombie. Le parent utilise l'appel système texttt{wait*} pour nettoyer le zombie du système. La famille d'appels système texttt{wait} inclut texttt{wait}, texttt{waitpid}, texttt{wait3}, texttt{wait4}. Par exemple :

begin{lstlisting}
pid_t wait4(pid_t, int *wstatus, int options, struct rusage *rusage);
end{lstlisting}

Le parent peut dormir/attendre que son enfant termine ou s'exécute en parallèle. L'appel texttt{wait*()} bloquera à moins que texttt{WNOHANG} ne soit donné dans les options. Pour approfondir, consultez texttt{man 2 wait}.

section{texttt{fork()} - Créer un processus fils}

La fonction texttt{fork()} initialise un nouveau PCB basé sur la valeur du parent et ajoute le PCB à la file d'attente exécutable. Ainsi, deux processus existent au même point d'exécution.

L'espace d'adressage du fils est une copie complète de celui du parent, avec une différence notable. La fonction texttt{fork()} retourne deux fois : une fois au parent avec un texttt{pid} supérieur à 0, et une fois au fils avec un texttt{pid} égal à 0.

begin{tcolorbox}[title={À retenir}]
La variable globale texttt{errno} contient le numéro d'erreur de la dernière appel système.
end{tcolorbox}

Quel est l'ordre d'impression ? Voici un exemple de code :

begin{lstlisting}
int main(int argc, char *argv[])
{
    int pid = fork();
    if (pid == 0) {
        // Enfant
        printf("parent=% d son=% dn", getppid(), getpid());
    }
    else if (pid > 0) {
        // Parent
        printf("parent=% d son=% dn", getpid(), pid);
    }
    else {
        // Afficher la chaine associee a errno
        perror("fork() failed");
    }
    return 0;
}
end{lstlisting}

end{document}

end{document}
