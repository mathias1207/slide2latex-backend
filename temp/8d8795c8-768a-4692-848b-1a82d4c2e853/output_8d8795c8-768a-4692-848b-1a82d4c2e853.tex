\documentclass[12pt]{article}
\usepackage[utf8]{inputenc}
\usepackage[T1]{fontenc}
\usepackage[french]{babel}

\usepackage{tcolorbox}
\usepackage{fontawesome5}
\usepackage{listings}
\usepackage{amsmath}
\usepackage{xcolor}
\usepackage{geometry}
\usepackage{textcomp}
\DeclareUnicodeCharacter{00A0}{~}
\DeclareUnicodeCharacter{200B}{}

% Configuration simplifiée pour éviter les erreurs avec \hss
\lstset{
    basicstyle=\ttfamily\small,
    breaklines=true,
    breakatwhitespace=false,
    keepspaces=true,
    numbers=left,
    numberstyle=\tiny\color{gray},
    showspaces=false,
    showstringspaces=false,
    showtabs=false,
    tabsize=2
}

% Configuration minimale de tcolorbox
\tcbuselibrary{most}

% Ajouter cette commande pour les échappements
\newcommand{\escapesym}[1]{\texttt{\textbackslash{}#1}}

% Configuration des marges
\geometry{margin=2.5cm}

\title{saea}
\author{}
\date{}

\begin{document}
\maketitle
\tableofcontents
\newpage

\section{Introduction}

\subsection{Objectif : Digitalisation}

Vous disposez d'un signal réel et souhaitez le représenter numériquement dans un ordinateur. Cela implique certaines contraintes : il existe un nombre fini de valeurs pour représenter l'ensemble du signal $N$. Chaque valeur ne peut pas prendre n'importe quelle valeur réelle, mais seulement l'une des $k$ valeurs quantifiées fixes, où $b = \log_2(k)$.

% FIGURE: Illustration de la digitalisation avec des images de la Mona Lisa

\begin{tcolorbox}[title={Vulgarisation simple}]
Imaginez que vous devez dessiner une image avec seulement quelques couleurs. Vous ne pouvez pas utiliser toutes les nuances possibles, mais seulement un ensemble limité de couleurs prédéfinies. C'est similaire à la digitalisation d'un signal.
\end{tcolorbox}

\newpage

\section{Échantillonnage}

\subsection{Concept de l'échantillonnage}

Le signal $\varphi(x)$ est décrit par ses valeurs pour tous les $x$. Cependant, représenter $\varphi$ nécessite une infinité de valeurs $\varphi(x)$. L'objectif est de trouver un ensemble fini de nombres $\hat{\varphi}_1, \ldots, \hat{\varphi}_N$ pour représenter approximativement $\varphi$.

\begin{align*}
\begin{pmatrix}
\hat{\varphi}_1 \\
\vdots \\
\hat{\varphi}_N
\end{pmatrix}
\end{align*}

\subsection{Stratégies d'échantillonnage}

Il n'existe pas d'unicité dans les stratégies d'échantillonnage et de reconstruction. Étant donné une stratégie de reconstruction et une définition de "meilleur", il s'agit de trouver les "meilleurs" $\hat{\varphi}_1, \ldots, \hat{\varphi}_N$.

% FIGURE: Illustration des stratégies d'échantillonnage

\subsection{Stratégie commune}

Une stratégie courante consiste à quadriller le domaine et à effectuer une reconstruction constante par morceaux sur chaque cellule de la grille. Avec une grille uniforme, cela correspond à un échantillonnage uniforme.

\newpage

\section{Critère $L^2$}

\subsection{Erreur Quadratique Moyenne (MSE)}

La MSE totale est définie par :

\begin{equation}
MSE = \int_0^1 (\varphi(x) - \hat{\varphi}(x))^2 dx
\end{equation}

Pour le cas général (non uniforme) :

\begin{equation}
MSE = \frac{1}{|I|} \int_J (\varphi(x) - \hat{\varphi}(x))^2 dx
\end{equation}

\subsection{Condition d'optimalité}

La condition d'optimalité est donnée par :

\begin{equation}
\frac{\partial}{\partial \hat{\varphi}_i} MSE = 0 \implies \hat{\varphi}_i = \frac{1}{\Delta} \int_{(i-1)\Delta}^{i\Delta} \varphi(x) dx
\end{equation}

Les échantillons optimaux sont les moyennes locales sur chaque intervalle d'échantillonnage.

\begin{tcolorbox}[colback=red!5!white, colframe=red!75!black, title={\faBookmark\hspace{0.5em}Fiche Récapitulative}]
  
\textbf{Objectif :} Résumer le concept principal de cette section.

\textbf{Principe central :} Utiliser l'erreur quadratique moyenne pour évaluer la qualité de l'échantillonnage.

\vspace{0.4em}
\textbf{Points essentiels :}  
- Calculer la MSE pour évaluer l'erreur
- Trouver les échantillons optimaux
- Utiliser les moyennes locales

\vspace{0.4em}
\textbf{Formules clés :}  
\begin{itemize}
\item $MSE = \int_0^1 (\varphi(x) - \hat{\varphi}(x))^2 dx$
\item $\hat{\varphi}_i = \frac{1}{\Delta} \int_{(i-1)\Delta}^{i\Delta} \varphi(x) dx$
\end{itemize}

\end{tcolorbox}

\newpage

\section{Critère $L^1$}

\subsection{Déviation Absolue Moyenne (MAD)}

La MAD totale est définie par :

\begin{equation}
MAD = \int_0^1 |\varphi(x) - \hat{\varphi}(x)| dx
\end{equation}

Pour le cas général (non uniforme) :

\begin{equation}
MAD = \frac{1}{|I|} \int_J |\varphi(x) - \hat{\varphi}(x)| dx
\end{equation}

\subsection{Condition d'optimalité}

La condition d'optimalité est donnée par :

\begin{equation}
\exists s, \frac{\partial}{\partial \hat{\varphi}_i} MAD = 0 \iff \exists s, \int_{I_i} \varphi(x) < \hat{\varphi}_i dx = \int_{I_i} \varphi(x) > \hat{\varphi}_i dx + s \int_{I_i} \varphi(x) = \hat{\varphi}_i dx
\end{equation}

Les échantillons optimaux sont les médianes locales sur chaque intervalle d'échantillonnage.

\begin{tcolorbox}[colback=red!5!white, colframe=red!75!black, title={\faBookmark\hspace{0.5em}Fiche Récapitulative}]
  
\textbf{Objectif :} Résumer le concept principal de cette section.

\textbf{Principe central :} Utiliser la déviation absolue moyenne pour évaluer la qualité de l'échantillonnage.

\vspace{0.4em}
\textbf{Points essentiels :}  
- Calculer la MAD pour évaluer l'erreur
- Trouver les échantillons optimaux
- Utiliser les médianes locales

\vspace{0.4em}
\textbf{Formules clés :}  
\begin{itemize}
\item $MAD = \int_0^1 |\varphi(x) - \hat{\varphi}(x)| dx$
\item Condition d'optimalité basée sur les médianes
\end{itemize}

\end{tcolorbox}

\newpage

\section{Exemples}

\subsection{Exemple 1}

Pour $\varphi(x) = x$, l'échantillonnage $L^2$ et $L^1$ donne des résultats différents sur les intervalles $I_i = \left[\frac{i-1}{N}, \frac{i}{N}\right)$.

\subsection{Exemple 2}

Pour $\varphi(x) = \begin{cases} 
1 & \text{si } x \in [0, \frac{1}{4}] \\
\frac{4}{3}(1-x) & \text{sinon}
\end{cases}$ avec $N = 2$ échantillons, les résultats de l'échantillonnage $L^2$ et $L^1$ sont comparés.

\newpage

\section{Échantillonnage dans le Monde Discret}

Il est possible de sous-échantillonner de $N$ à $M$ en utilisant des moyennes sur les intervalles discrets.

% FIGURE: Illustration du sous-échantillonnage

\begin{tcolorbox}[title={Vulgarisation simple}]
Imaginez que vous avez une longue liste de nombres et que vous voulez la réduire. Vous pouvez prendre la moyenne de chaque groupe de nombres pour obtenir une version plus courte mais représentative de la liste originale.
\end{tcolorbox}

\end{document}