\documentclass[12pt]{article}
\usepackage[utf8]{inputenc}
\usepackage[T1]{fontenc}
\usepackage[french]{babel}

\usepackage{tcolorbox}
\usepackage{fontawesome5}
\usepackage{listings}
\usepackage{amsmath}
\usepackage{xcolor}
\usepackage{geometry}
\usepackage{textcomp}
\DeclareUnicodeCharacter{00A0}{~}
\DeclareUnicodeCharacter{200B}{}

% Configuration simplifiée pour éviter les erreurs avec \hss
\lstset{
    basicstyle=\ttfamily\small,
    breaklines=true,
    breakatwhitespace=false,
    keepspaces=true,
    numbers=left,
    numberstyle=\tiny\color{gray},
    showspaces=false,
    showstringspaces=false,
    showtabs=false,
    tabsize=2
}

% Configuration minimale de tcolorbox
\tcbuselibrary{most}

% Ajouter cette commande pour les échappements
\newcommand{\escapesym}[1]{\texttt{\textbackslash{}#1}}

% Configuration des marges
\geometry{margin=2.5cm}

\title{datta}
\author{}
\date{}

\begin{document}
\maketitle
\tableofcontents
\newpage

\section{Introduction}

\sub\section{Objectif : Digitalisation}

L'objectif principal de cette \section est de comprendre comment un signal réel peut être adapté pour être traité par un ordinateur. Cela implique de prendre en compte certaines contraintes. Le signal doit être représenté par un nombre fini de valeurs, noté \(N\). De plus, chaque valeur ne peut pas prendre n'importe quelle valeur réelle, mais doit être l'une des \(k\) valeurs quantifiées fixes, où \(b = \log_2(k)\).

\begin{tcolorbox}[title={Intuition}]
La digitalisation d'un signal peut être comparée à la transformation d'une image en pixels. Un signal continu est comme une image haute résolution, tandis qu'un signal échantillonné et quantifié est comme une image pixelisée.
\end{tcolorbox}

% FIGURE: Illustration de la digitalisation avec la Mona Lisa

\sub\section{Processus de Digitalisation}

Le processus de digitalisation peut être visualisé en trois étapes : le signal continu, le signal échantillonné, et le signal digitalisé (échantillonné et quantifié).

% FIGURE: Graphiques montrant le processus de digitalisation

\begin{tcolorbox}[title={Vulgarisation simple}]
Imaginez que vous devez dessiner une courbe complexe avec seulement quelques points. La digitalisation consiste à choisir ces points de manière à ce que la courbe soit approximativement représentée.
\end{tcolorbox}

\newpage

\section{Échantillonnage}

\sub\section{Concept d'Échantillonnage}

L'échantillonnage consiste à trouver un ensemble fini de nombres \(\hat{\varphi}_1, \ldots, \hat{\varphi}_N\) pour représenter approximativement une fonction \(\varphi(x)\). Cette fonction est décrite par ses valeurs pour tous les \(x\), nécessitant une infinité de valeurs pour une représentation exacte.

\begin{tcolorbox}[title={Intuition}]
L'échantillonnage est comme prendre des photos à intervalles réguliers pour capturer un événement en continu. Plus les photos sont fréquentes, plus la représentation est fidèle.
\end{tcolorbox}

% FIGURE: Illustration de l'échantillonnage

\sub\section{Stratégies d'Échantillonnage}

Il n'existe pas d'unicité dans les stratégies d'échantillonnage et de reconstruction. Une stratégie courante est de griller le domaine et de reconstruire de manière constante par morceaux.

% FIGURE: Stratégie d'échantillonnage avec grille uniforme

\begin{tcolorbox}[title={Fiche Récapitulative}]
\textbf{Objectif :} Comprendre l'échantillonnage et la reconstruction de signaux.

\textbf{Principe central :} Utiliser un ensemble fini de valeurs pour représenter un signal continu.

\vspace{0.4em}
\textbf{Points essentiels :}  
- L'échantillonnage consiste à choisir des points représentatifs
- La reconstruction peut être constante par morceaux
- La grille uniforme est une stratégie courante

\vspace{0.4em}
\textbf{À retenir :}  
- Utiliser des tirets pour les listes au lieu de \begin{itemize}
- Insérer des \vspace{0.4em} entre les \sections pour l'espacement
- Mettre en gras (\textbf) les titres des \sections
\end{tcolorbox}

\newpage

\section{Critère \(L^2\)}

\sub\section{Erreur Quadratique Moyenne}

Le critère \(L^2\) utilise l'erreur quadratique moyenne (MSE) pour évaluer la qualité de l'échantillonnage. La MSE totale est définie par :

\begin{equation}
MSE = \int_0^1 (\varphi(x) - \hat{\varphi}(x))^2 dx
\end{equation}

La MSE locale est calculée sur chaque intervalle \(I_i\).

\begin{tcolorbox}[title={Vulgarisation simple}]
L'erreur quadratique moyenne est comme mesurer la différence entre la courbe originale et la courbe reconstruite, en prenant la moyenne des carrés de ces différences.
\end{tcolorbox}

% FIGURE: Graphique illustrant la MSE

\sub\section{Condition d'Optimalité}

La condition d'optimalité pour minimiser la MSE est que les échantillons optimaux sont les moyennes locales sur chaque intervalle d'échantillonnage.

\begin{equation}
\frac{\partial}{\partial \hat{\varphi}_i} MSE = 0 \implies \hat{\varphi}_i = \frac{1}{\Delta} \int_{(i-1)\Delta}^{i\Delta} \varphi(x) dx
\end{equation}

\begin{tcolorbox}[title={Fiche Récapitulative}]
\textbf{Objectif :} Minimiser l'erreur quadratique moyenne dans l'échantillonnage.

\textbf{Principe central :} Utiliser la MSE pour évaluer et optimiser l'échantillonnage.

\vspace{0.4em}
\textbf{Points essentiels :}  
- La MSE mesure la différence quadratique entre le signal original et échantillonné
- Les échantillons optimaux sont les moyennes locales
- La condition d'optimalité guide le choix des échantillons

\vspace{0.4em}
\textbf{À retenir :}  
- Utiliser des tirets pour les listes au lieu de \begin{itemize}
- Insérer des \vspace{0.4em} entre les \sections pour l'espacement
- Mettre en gras (\textbf) les titres des \sections
\end{tcolorbox}

\newpage

\section{Critère \(L^1\)}

\sub\section{Déviation Absolue Moyenne}

Le critère \(L^1\) utilise la déviation absolue moyenne (MAD) pour évaluer l'échantillonnage. La MAD totale est définie par :

\begin{equation}
MAD = \int_0^1 |\varphi(x) - \hat{\varphi}(x)| dx
\end{equation}

\begin{tcolorbox}[title={Intuition}]
La MAD est comme mesurer la différence absolue entre la courbe originale et la courbe reconstruite, en prenant la moyenne de ces différences.
\end{tcolorbox}

% FIGURE: Graphique illustrant la MAD

\sub\section{Condition d'Optimalité}

La condition d'optimalité pour minimiser la MAD implique l'utilisation de sous-différentiels en raison de la non-différentiabilité de la fonction absolue à zéro.

\begin{equation}
\exists s, \frac{\partial}{\partial \hat{\varphi}_i} MAD = 0 \iff \exists s, \int_{I_i} \varphi(x) < \hat{\varphi}_i \, dx = \int_{I_i} \varphi(x) > \hat{\varphi}_i \, dx + s \int_{I_i} \varphi(x) = \hat{\varphi}_i \, dx
\end{equation}

\begin{tcolorbox}[title={Fiche Récapitulative}]
\textbf{Objectif :} Minimiser la déviation absolue moyenne dans l'échantillonnage.

\textbf{Principe central :} Utiliser la MAD pour évaluer et optimiser l'échantillonnage.

\vspace{0.4em}
\textbf{Points essentiels :}  
- La MAD mesure la différence absolue entre le signal original et échantillonné
- Les échantillons optimaux sont les médianes locales
- La condition d'optimalité utilise des sous-différentiels

\vspace{0.4em}
\textbf{À retenir :}  
- Utiliser des tirets pour les listes au lieu de \begin{itemize}
- Insérer des \vspace{0.4em} entre les \sections pour l'espacement
- Mettre en gras (\textbf) les titres des \sections
\end{tcolorbox}

\newpage

\section{Exemples}

\sub\section{Exemple 1}

Considérons la fonction \(\varphi(x) = x\). Pour l'échantillonnage \(L^2\), sur l'intervalle \(I_i = \left[\frac{i-1}{N}, \frac{i}{N}\right)\), nous avons :

\begin{equation}
\hat{\varphi}_i^{MSE} = \frac{1}{\Delta} \int_{\frac{i-1}{N}}^{\frac{i}{N}} x \, dx = \Delta \left(i - \frac{1}{2}\right)
\end{equation}

Pour l'échantillonnage \(L^1\), nous avons :

\begin{equation}
\hat{\varphi}_i^{MAD} = MEDIAN(\{x \mid x \in [(i-1)\Delta, i\Delta]\}) = \Delta \left(i - \frac{1}{2}\right)
\end{equation}

\begin{tcolorbox}[title={Vulgarisation simple}]
Dans cet exemple, les échantillons optimaux pour les critères \(L^2\) et \(L^1\) sont les mêmes, car la fonction est linéaire.
\end{tcolorbox}

% FIGURE: Graphique illustrant l'exemple 1

\sub\section{Exemple 2}

Pour la fonction \(\varphi(x)\) définie par :

\[
\varphi(x) = 
\begin{cases} 
1 & \text{si } x \in [0, \frac{1}{4}] \\
\frac{4}{3}(1-x) & \text{sinon}
\end{cases}
\]

Avec \(N = 2\) échantillons, les valeurs optimales pour \(L^2\) et \(L^1\) sont calculées.

\begin{tcolorbox}[title={Fiche Récapitulative}]
\textbf{Objectif :} Illustrer l'application des critères \(L^2\) et \(L^1\) à des exemples concrets.

\textbf{Principe central :} Comparer les résultats des deux critères sur des fonctions spécifiques.

\vspace{0.4em}
\textbf{Points essentiels :}  
- Les critères \(L^2\) et \(L^1\) peuvent donner des résultats différents
- Les échantillons optimaux dépendent de la forme de la fonction
- Les exemples illustrent l'application pratique des concepts théoriques

\vspace{0.4em}
\textbf{À retenir :}  
- Utiliser des tirets pour les listes au lieu de \begin{itemize}
- Insérer des \vspace{0.4em} entre les \sections pour l'espacement
- Mettre en gras (\textbf) les titres des \sections
\end{tcolorbox}

\newpage

\section{Échantillonnage dans le Monde Discret}

\sub\section{Sous-échantillonnage}

Le sous-échantillonnage est possible en passant de \(N\) à \(M\) échantillons, où \(N = DM\). Les échantillons optimaux sont les moyennes sur les intervalles discrets.

% FIGURE: Illustration du sous-échantillonnage

\begin{tcolorbox}[title={Intuition}]
Le sous-échantillonnage est comme réduire la résolution d'une image en conservant les informations essentielles.
\end{tcolorbox}

\begin{tcolorbox}[title={Fiche Récapitulative}]
\textbf{Objectif :} Comprendre le sous-échantillonnage dans un contexte discret.

\textbf{Principe central :} Réduire le nombre d'échantillons tout en préservant l'information.

\vspace{0.4em}
\textbf{Points essentiels :}  
- Le sous-échantillonnage réduit la taille des données
- Les échantillons sont les moyennes des intervalles
- Cela permet une représentation efficace dans le monde discret

\vspace{0.4em}
\textbf{À retenir :}  
- Utiliser des tirets pour les listes au lieu de \begin{itemize}
- Insérer des \vspace{0.4em} entre les \sections pour l'espacement
- Mettre en gras (\textbf) les titres des \sections
\end{tcolorbox}
\end{itemize}}}}}

\end{document}