\documentclass[12pt]{report}
\usepackage[utf8]{inputenc}
\usepackage[hebrew]{babel}

\usepackage{polyglossia}
\setmainlanguage{hebrew}
\setotherlanguage{english}
\newfontfamily\hebrewfont[Script=Hebrew]{Arial}
\newfontfamily\englishfont[Script=Latin]{Arial}

\usepackage{tcolorbox}
\usepackage{fontawesome5}
\usepackage{listings}
\usepackage{algorithm2e}
\usepackage{amsmath,amsfonts,amssymb,graphicx,float}
\usepackage{xcolor}

% Configuration des listings
\lstset{
    basicstyle=\ttfamily\small,
    breaklines=true,
    frame=single,
    numbers=left,
    numberstyle=\tiny,
    numbersep=5pt,
    showstringspaces=false,
    language=C,
    keepspaces=true,
    columns=flexible,
    literate={*}{\textasteriskcentered}1
}

% Configuration des tcolorbox
\tcbuselibrary{most}
\newtcolorbox{mybox}[1][]{
    colback=white,
    colframe=black,
    coltitle=black,
    title=#1,
    fonttitle=\bfseries
}

\graphicspath{{./images/}}
\title{os}
\author{}

\begin{document}
\maketitle
\tableofcontents
\newpage

\section{Introduction}
% Cours : os
  \section{יצירת תהליך וסיום}  \begin{itemize}  \item תהליך אחד (ה\"הורה\") יכול ליצור תהליך אחר (ה\"ילד\").  \item ישנה הקצאה ואתחול של PCB חדש.  \item שיעורי בית: הפעלה של 'ps auxwww' ב-shell; PPID הוא ה-PID של ההורה.  \end{itemize}    \section{ב-POSIX, התהליך הילד מיירש את רוב התכונות של ההורה}  \begin{itemize}  \item UID, קבצים פתוחים (יש לסגור אם אינם נדרשים; למה?), cwd, וכו'.  \end{itemize}    \section{בעת ביצוע, ה-PCB מעביר בין תורים שונים}  \begin{itemize}  \item לפי גרף שינוי המצב.  \item תורים: runnable, sleep/wait for event i (i=1,2,3…).  \end{itemize}    \section{לאחר שתהליך מת (exit()s / interrupted), הוא הופך לזומבי}  \begin{itemize}  \item ההורה משתמש בקריאת מערכת wait* כדי לנקות את הזומבי מהמערכת (למה?).  \item משפחת קריאות המערכת wait: wait, waitpid, waitid, wait3, wait4; לדוגמה:  \end{itemize}    \section{ההורה יכול לישון/לחכות לילד שלו לסיים או לרוץ במקביל}  \begin{itemize}  \item wait*() יחסום אלא אם כן ניתן WNOHANG ב-'options'.  \item שיעורי בית: קרא 'man 2 wait'.  \end{itemize}  
  \begin{lstlisting}[language=C]  pid_t wait4(pid_t, int *wstatus, int options, struct rusage *rusage);  \end{lstlisting}  
  \begin{tcolorbox}[colback=yellow!5, colframe=yellow!80!black, title={\faBookmark נקודות מפתח}]  תהליך יכול ליצור תהליך אחר, והתהליך ה\"ילד\" מיירש את רוב התכונות של התהליך ה\"הורה\". במהלך הביצוע, ה-PCB מתנהל בין תורים שונים. לאחר שתהליך מת, הוא הופך לזומבי, וההורה צריך לנקות אותו מהמערכת באמצעות קריאת מערכת wait*. ההורה יכול להמתין לילד שלו לסיים או לרוץ במקביל.  \end{tcolorbox}  

int main(int argc, char *argv[])
{
  int pid = fork();
  if( pid==0 ) { 
   //
   // child
      //
      printf(“parent=%d son=%d\n”,
             getppid(), getpid());
  }
  else if( pid > 0 ) {
      //
      // parent
      //
      printf(“parent=%d son=%d\n”,
             getpid(), pid);
  }
  else { // print string associated
         // with errno   
      perror(“fork() failed”); 
  }
  return 0;
}
• fork() initializes a new PCB
– Based on parent’s value
– PCB added to runnable queue
• Now there are 2 processes
– At same execution point
• Child’s new address space 
– Complete copy of parent’s 
space, with one difference…
• fork() returns twice
– At the parent, with pid>0
– At the child, with pid=0
• What’s the printing order?
• ‘errno’ – a global variable
– Holds error num of last syscall
OS (234123) - processes & signals
8
fork() – spawn a child process


\end{document}