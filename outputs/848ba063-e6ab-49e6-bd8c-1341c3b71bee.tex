% !TEX program = xelatex
\documentclass[12pt]{report}
\usepackage[utf8]{inputenc}
\usepackage[hebrew]{babel}

\usepackage{fontspec}
\usepackage{polyglossia}
\setmainlanguage{hebrew}
\setotherlanguage{english}
\newfontfamily\hebrewfont[Script=Hebrew]{Noto Sans Hebrew}
\newfontfamily\englishfont[Script=Latin]{Noto Sans}
\setmainfont{Noto Sans}
\setmonofont{Noto Sans Mono}

\usepackage{tcolorbox}
\usepackage{fontawesome5}
\usepackage{listings}
\usepackage{algorithm2e}
\usepackage{amsmath,amsfonts,amssymb,graphicx,float}
\usepackage{xcolor}

% Configuration des listings
\lstset{
    basicstyle=\ttfamily\small,
    breaklines=true,
    frame=single,
    numbers=left,
    numberstyle=\tiny,
    numbersep=5pt,
    showstringspaces=false,
    language=C,
    keepspaces=true,
    columns=flexible,
    literate={*}{\textasteriskcentered}1
}

% Configuration des tcolorbox
\tcbuselibrary{most}
\newtcolorbox{mybox}[1][]{
    colback=white,
    colframe=black,
    coltitle=black,
    title=#1,
    fonttitle=\bfseries
}

\graphicspath{{./images/}}
\title{os}
\author{}

\begin{document}
\maketitle
\tableofcontents
\newpage

\section{Introduction}
% Cours : os
  \section{יצירה וסיום תהליך}  \begin{itemize}  \item תהליך אחד (ה\"הורה\") יכול ליצור תהליך אחר (ה\"ילד\")  \item לוח הבקרה של התהליך (PCB) מוקצה ומאותחל  \item שיעורי בית: הרץ 'ps auxwww' ב-shell; PPID הוא ה-PID של ההורה  \end{itemize}    \subsection{יורשת תהליך ב-POSIX}  \begin{itemize}  \item ב-POSIX, התהליך ה\"ילד\" יורש את רוב מאפייני ההורה  \item UID, קבצים פתוחים (אמורים להיסגר אם אינם נדרשים; למה?), cwd, וכו'.  \end{itemize}    \subsection{ביצוע תהליך}  \begin{itemize}  \item במהלך הביצוע, PCB מעביר בין תורים שונים  \item לפי גרף שינוי המצב  \item תורים: runnable, sleep/wait for event i (i=1,2,3…)  \end{itemize}    \section{סיום תהליך}  \begin{itemize}  \item לאחר שתהליך מת (exit()s / interrupted), הוא הופך לזומבי  \item ההורה משתמש ב- syscall של wait* כדי לנקות את הזומבי מהמערכת (למה?)  \item משפחת ה- syscall של wait: wait, waitpid, waitid, wait3, wait4; לדוגמה:  \end{itemize}    \section{המתנה / ריצה מקבילית של ההורה}  \begin{itemize}  \item ההורה יכול לישון / להמתין לילד שלו לסיים או לרוץ במקביל  \item wait*() יחסום אלא אם כן WNOHANG ניתן ב- 'options'  \item שיעורי בית: קרא 'man 2 wait'  \end{itemize}  
  \begin{lstlisting}[language=C]  pid_t wait4(pid_t, int *wstatus, int options, struct rusage *rusage);  \end{lstlisting}  
  \begin{tcolorbox}[colback=yellow!5, colframe=yellow!80!black, title={\faBookmark נקודות מפתח}]  במערכת הפעלה, תהליך אחד יכול ליצור תהליך אחר. התהליך ה\"ילד\" יורש את רוב מאפייני ההורה. במהלך הביצוע, לוח הבקרה של התהליך (PCB) מעביר בין תורים שונים. לאחר שתהליך מת, הוא הופך לזומבי.  \end{tcolorbox}  

\section{יצירת תהליך ילד באמצעות fork()} 
 בעזרת הפקודה fork() אפשר ליצור תהליך ילד. הפקודה מאתחלת PCB חדש בהתאם לערך של ההורה, ומוסיפה אותו לתור הביצוע. לאחר מכן, ישנם שני תהליכים שנמצאים באותו נקודת ביצוע. מרחב הכתובות של הילד הוא העתק מוחלט של מרחב הכתובות של ההורה, עם הבדל אחד... 

 \subsection{החזרת ערך של fork()} 
 הפקודה fork() מחזירה פעמיים: בהורה, עם pid>0, ובילד, עם pid=0. 

 \subsection{סדר ההדפסה} 
 מהו סדר ההדפסה? 

 \subsection{משתנה הגלובלי 'errno'} 
 'errno' הוא משתנה גלובלי שמאחסן את מספר השגיאה של ה- syscall האחרון.
\begin{lstlisting}[language=C] 
int main(int argc, char *argv[]) 
{ 
  int pid = fork(); 
  if( pid==0 ) { 
   // 
   // child 
      // 
      printf(“parent=%d son=%d\n”, 
             getppid(), getpid()); 
  } 
  else if( pid > 0 ) { 
      // 
      // parent 
      // 
      printf(“parent=%d son=%d\n”, 
             getpid(), pid); 
  } 
  else { // print string associated 
         // with errno   
      perror(“fork() failed”); 
  } 
  return 0; 
} 
\end{lstlisting}
\begin{tcolorbox}[colback=yellow!5, colframe=yellow!80!black, title={\faBookmark נקודות מפתח}] 
\begin{itemize} 
\item fork() מאתחלת PCB חדש בהתאם לערך של ההורה ומוסיפה אותו לתור הביצוע. 
\item לאחר הפקודה fork(), ישנם שני תהליכים שנמצאים באותו נקודת ביצוע. 
\item מרחב הכתובות של הילד הוא העתק מוחלט של מרחב הכתובות של ההורה. 
\item הפקודה fork() מחזירה פעמיים: בהורה, עם pid>0, ובילד, עם pid=0. 
\item 'errno' הוא משתנה גלובלי שמאחסן את מספר השגיאה של ה- syscall האחרון. 
\end{itemize} 
\end{tcolorbox}


\end{document}