% !TEX program = xelatex
\documentclass[12pt]{report}
\usepackage[utf8]{inputenc}
\usepackage[hebrew]{babel}

\usepackage{fontspec}
\usepackage{polyglossia}
\setmainlanguage{hebrew}
\setotherlanguage{english}
\newfontfamily\hebrewfont[Script=Hebrew]{Noto Sans Hebrew}
\newfontfamily\englishfont[Script=Latin]{Noto Sans}
\setmainfont{Noto Sans}
\setmonofont{Noto Sans Mono}

\usepackage{tcolorbox}
\usepackage{fontawesome5}
\usepackage{listings}
\usepackage{algorithm2e}
\usepackage{amsmath,amsfonts,amssymb,graphicx,float}
\usepackage{xcolor}

% Configuration des listings
\lstset{
    basicstyle=\ttfamily\small,
    breaklines=true,
    frame=single,
    numbers=left,
    numberstyle=\tiny,
    numbersep=5pt,
    showstringspaces=false,
    language=C,
    keepspaces=true,
    columns=flexible,
    literate={*}{\textasteriskcentered}1
}

% Configuration des tcolorbox
\tcbuselibrary{most}
\newtcolorbox{mybox}[1][]{
    colback=white,
    colframe=black,
    coltitle=black,
    title=#1,
    fonttitle=\bfseries
}

\graphicspath{{./images/}}
\title{os}
\author{}

\begin{document}
\maketitle
\tableofcontents
\newpage

\section{Introduction}
% Cours : os
  \section{יצירת תהליך וסיום}  \begin{itemize}  \item תהליך אחד (ה'הורה') יכול ליצור אחר (ה'ילד')  \item לוח בקרה חדש מוקצה ומאותחל  \item שיעורי בית: הרץ 'ps auxwww' ב-shell; PPID הוא PID של ההורה  \end{itemize}    \subsection{POSIX}  \begin{itemize}  \item ב-POSIX, תהליך הילד מיירש את רוב התכונות של ההורה  \item UID, קבצים פתוחים (צריך להיסגר אם אינם נדרשים; למה?), cwd, וכו'.  \end{itemize}    \subsection{ביצוע תהליך}  \begin{itemize}  \item בעת הביצוע, לוח הבקרה מעביר בין תורים שונים  \item לפי גרף שינוי המצב  \item תורים: runnable, sleep/wait for event i (i=1,2,3…)  \end{itemize}    \section{סיום תהליך}  \begin{itemize}  \item לאחר שתהליך מת (exit()s / interrupted), הוא הופך לזומבי  \item ההורה משתמש בקריאת מערכת wait* כדי לנקות את הזומבי מהמערכת (למה?)  \item משפחת קריאות המערכת של wait: wait, waitpid, waitid, wait3, wait4; לדוגמה:  \end{itemize}    \section{המתנה / המשך ריצה של ההורה}  \begin{itemize}  \item ההורה יכול להמתין / להמשיך לרוץ במקביל לילד שלו  \item wait*() יחסום אלא אם כן WNOHANG ניתן ב-'options'  \item שיעורי בית: קרא 'man 2 wait'  \end{itemize}  
  \begin{lstlisting}[language=C]  pid_t wait4(pid_t, int *wstatus, int options, struct rusage *rusage);  \end{lstlisting}  

\section{יצירת תהליך חדש עם fork()} 
 פונקציית fork() מאתחלת לוח בקרה חדש (PCB) על פי הערך של ההורה. לאחר מכן, ה-PCB מתווסף לתור הביצועים, וכעת ישנם שני תהליכים שנמצאים באותו נקודת ביצוע. 

 \begin{itemize} 
 \item מרחב הכתובת החדש של הילד הוא עותק מלא של מרחב הכתובת של ההורה, עם הבדל אחד... 
 \item fork() מחזירה פעמיים: בהורה, עם pid>0; בילד, עם pid=0. 
 \item מהו סדר ההדפסה? 
 \item 'errno' - משתנה גלובלי שמחזיק את מספר השגיאה של הקריאה למערכת האחרונה. 
 \end{itemize}
\begin{lstlisting}[language=C] 
int main(int argc, char *argv[]) 
{ 
  int pid = fork(); 
  if( pid==0 ) { 
   // 
   // child 
      // 
      printf("parent=%d son=%d\n", 
             getppid(), getpid()); 
  } 
  else if( pid > 0 ) { 
      // 
      // parent 
      // 
      printf("parent=%d son=%d\n", 
             getpid(), pid); 
  } 
  else { // print string associated 
         // with errno   
      perror("fork() failed");  
  } 
  return 0; 
} 
\end{lstlisting}
\begin{tcolorbox}[colback=yellow!5, colframe=yellow!80!black, title={\faBookmark נקודות מפתח}] 
פונקציית fork() מאתחלת לוח בקרה חדש (PCB) על פי הערך של ההורה. לאחר מכן, ה-PCB מתווסף לתור הביצועים, וכעת ישנם שני תהליכים שנמצאים באותו נקודת ביצוע. 
\end{tcolorbox}
\begin{tcolorbox}[colback=green!5, colframe=green!75!black, title={\faLightbulb אינטואיציה}] 
פונקציית fork() מאפשרת לנו ליצור תהליך חדש שהוא עותק של התהליך הנוכחי. זה מאפשר לנו להריץ קוד במקביל, ולשתף מידע בין התהליכים. 
\end{tcolorbox}
\begin{tcolorbox}[colback=blue!5, colframe=blue!75!black, title={\faLightbulb הסבר פשוט}] 
פונקציית fork() היא כמו כפתור שמפעיל מכונת העתקה של התהליך הנוכחי. היא מעתיקה את כל המידע של התהליך הנוכחי לתהליך חדש, ומאפשרת לשני התהליכים להמשיך לרוץ במקביל. 
\end{tcolorbox}


\end{document}