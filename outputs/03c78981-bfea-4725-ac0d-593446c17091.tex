% !TEX program = xelatex
\documentclass[12pt]{report}
\usepackage[utf8]{inputenc}
\usepackage[hebrew]{babel}

\usepackage{fontspec}
\usepackage{polyglossia}
\setmainlanguage{hebrew}
\setotherlanguage{english}
\newfontfamily\hebrewfont[Script=Hebrew]{Noto Sans Hebrew}
\newfontfamily\englishfont[Script=Latin]{Noto Sans}
\setmainfont{Noto Sans}
\setmonofont{Noto Sans Mono}

\usepackage{tcolorbox}
\usepackage{fontawesome5}
\usepackage{listings}
\usepackage{algorithm2e}
\usepackage{amsmath,amsfonts,amssymb,graphicx,float}
\usepackage{xcolor}

% Configuration des listings
\lstset{
    basicstyle=\ttfamily\small,
    breaklines=true,
    frame=single,
    numbers=left,
    numberstyle=\tiny,
    numbersep=5pt,
    showstringspaces=false,
    language=C,
    keepspaces=true,
    columns=flexible,
    literate={*}{\textasteriskcentered}1
}

% Configuration des tcolorbox
\tcbuselibrary{most}
\newtcolorbox{mybox}[1][]{
    colback=white,
    colframe=black,
    coltitle=black,
    title=#1,
    fonttitle=\bfseries
}

\graphicspath{{./images/}}
\title{os}
\author{}

\begin{document}
\maketitle
\tableofcontents
\newpage

\section{Introduction}
% Cours : os
  \section{יצירת תהליך וסיום}    \subsection{יצירת תהליך}  \begin{itemize}  \item תהליך אחד (ה"הורה") יכול ליצור תהליך אחר (ה"ילד").  \item נוצר ומאותחל לוח PCB חדש.  \item שיעורי בית: הפעל 'ps auxwww' ב-shell; PPID הוא ה-PID של ההורה.  \end{itemize}    \subsection{ירושה ב-POSIX}  \begin{itemize}  \item ב-POSIX, התהליך הילד מוריש את רוב מאפייני ההורה.  \item UID, קבצים פתוחים (צריך לסגור אם אינם נדרשים; למה?), cwd, וכו'.  \end{itemize}    \subsection{ביצוע תהליך}  \begin{itemize}  \item במהלך הביצוע, PCB מתקדם בין תורים שונים.  \item לפי גרף שינוי מצב.  \item תורים: runnable, sleep/wait for event i (i=1,2,3…).  \end{itemize}    \section{סיום תהליך}    \subsection{תהליך זומבי}  \begin{itemize}  \item לאחר שתהליך מת (exit()s / interrupted), הוא הופך לזומבי.  \item ההורה משתמש בקריאת מערכת wait* כדי למחוק את הזומבי מהמערכת (למה?).  \end{itemize}    \subsection{קריאות מערכת wait}  \begin{itemize}  \item משפחת קריאות המערכת wait: wait, waitpid, waitid, wait3, wait4; לדוגמה:  \end{itemize}    \subsection{המתנה של ההורה}  \begin{itemize}  \item ההורה יכול לישון/להמתין לילד שלו לסיים או לרוץ במקביל.  \item wait*() תחסום אלא אם כן WNOHANG ניתן ב-'options'.  \item שיעורי בית: קרא 'man 2 wait'.  \end{itemize}  
  \begin{lstlisting}[language=C]  pid_t wait4(pid_t, int *wstatus, int options, struct rusage *rusage);  \end{lstlisting}  

\section{יצירת תהליך עם fork()} 

 בפונקציה fork() מתבצעת יצירה של תהליך חדש. זה מתבצע על ידי הפעלת פונקציה fork() שמקבלת כפרמטרים את הארגומנטים של התהליך. הפונקציה מחזירה את ה-PID של התהליך החדש או 0 אם הפונקציה נקראה מתוך התהליך החדש. 

 במקרה של שגיאה, הפונקציה מחזירה -1 ומעדכנת את המשתנה הגלובלי errno. הפונקציה perror() משמשת להדפסת הודעת השגיאה המתאימה לערך של errno. 

 עכשיו ישנם שני תהליכים שנמצאים באותו נקודת ביצוע, אך עם מרחבי כתובות שונים. מרחב הכתובות של התהליך החדש הוא העתק של מרחב הכתובות של התהליך ההורה, עם הבדל אחד - הערך שהפונקציה fork() מחזירה בכל אחד מהתהליכים. 

 מהו סדר ההדפסה? זה יכול להשתנות מהרצה להרצה, כיוון שלא ניתן לדעת מראש איזה מהתהליכים ירוץ קודם. 

\begin{lstlisting}[language=C]
int main(int argc, char *argv[])
{
  int pid = fork();
  if( pid==0 ) { 
   //
   // child
      //
      printf("parent=%d son=%d\n",
             getppid(), getpid());
  }
  else if( pid > 0 ) {
      //
      // parent
      //
      printf("parent=%d son=%d\n",
             getpid(), pid);
  }
  else { // print string associated
         // with errno   
      perror("fork() failed"); 
  }
  return 0;
}
\end{lstlisting}
\begin{tcolorbox}[colback=yellow!5, colframe=yellow!80!black, title={\faBookmark נקודות מפתח}]
הפונקציה fork() מאתחלת לוח PCB חדש, המבוסס על הערך של ההורה. לאחר מכן, ה-PCB מתווסף לתור הרצה. כעת ישנם שני תהליכים, שנמצאים באותו נקודת ביצוע. מרחב הכתובות החדש של הילד הוא עותק מלא של מרחב הכתובות של ההורה, עם הבדל אחד - הערך שהפונקציה fork() מחזירה בכל אחד מהתהליכים.
\end{tcolorbox}


\end{document}