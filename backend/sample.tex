\documentclass[12pt]{article}
\usepackage[utf8]{inputenc}
\usepackage[T1]{fontenc}
\usepackage[french]{babel}
\usepackage{tcolorbox}
\usepackage{fontawesome5}
\usepackage{listings}
\usepackage{amsmath}
\usepackage{xcolor}
\usepackage{geometry}
\usepackage{textcomp}

% Configuration simplifiée des listings
\lstset{
    basicstyle=\ttfamily\small,
    breaklines=true,
    breakatwhitespace=false,
    keepspaces=true,
    numbers=left,
    numberstyle=\tiny\color{gray},
    showspaces=false,
    showstringspaces=false,
    showtabs=false,
    tabsize=2
}

% Configuration de tcolorbox
\tcbuselibrary{most}

% Configuration des marges
\geometry{margin=2.5cm}

\title{Document de test}
\author{}
\date{}

\begin{document}
\maketitle

\section{Introduction}
Ceci est un document de test pour vérifier que la compilation LaTeX fonctionne correctement.

\begin{tcolorbox}[title={Test de boîte}]
Contenu de la boîte de test.
\end{tcolorbox}

\section{Formules mathématiques}
\begin{equation}
E = mc^2
\end{equation}

\section{Code}
\begin{lstlisting}
// Test de code
function test() {
    console.log("Hello, world!");
}
\end{lstlisting}

\end{document} 